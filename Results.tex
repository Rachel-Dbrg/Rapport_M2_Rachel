%!TEX root = ./Structure_rapport_final.tex

\subsection{Factorial Analysis of Mixed Data}

The first 5 FAMD components explained 80\% of the total variation (Table \ref{table:cont_abs}). The first principal component (PC1) (30.5\% of explained variance) is strongly correlated with the shape of the body and of the head with traits such as \emph{body depth} (absolute contribution $a = 11.7$), \emph{lower jaw length} ($a=11.5$), \emph{oral gape axis} ($a=11.1$), \emph{orbital length} ($a = 8.35$) and \emph{head length} ($a = 8.21$), thus, mostly to feeding and locomotion (Table~\ref{table:cont_abs}). Along this horizontal axis, all traits have increasing values towards the right, while mouth orientation is shifting upwards (see Figure~\ref{fig:corr_circ_12} for details). Three groups of fish can be distinguished (Figure~\ref{fig:famd12}). For low PC1 values, species have a low transversal shape (meaning that both \texttt{bd} and \texttt{bw} have similar values, consistent with an elongated body), short lower jaw, small oral gape surface and a terminal oral axis, such as \textit{A. risso}, \textit{S. boa} and \textit{S. beanii}. On the right of this axis, species display high body depth values suggesting a laterally compressed form and a superior-oriented oral axis, such as \textit{A. olfersii}. 

The second axis (21.6\% of explained variance) is correlated to feeding behavior, through food prospection with \emph{pectoral fin insertion} ($a = 11.4$), \emph{eye size} ($a = 9$), \emph{pectoral fin position} ($a = 7.1$) and food acquisition through \emph{gill raker type} $(a = 17.8$) and \emph{oral gape axis} ($a = 12.3$) (Table \ref{table:cont_abs}). This vertical axis separates the three species on the left, \textit{A. risso} and \textit{S. boa} display higher values of pectoral fin insertion, meaning that the fin is inserted further on the body than \textit{S. boa}, while the fin position on the body depth is lower. As \textit{S. beanii} does not have pectoral fin, the same analysis has been run without traits referring to pectoral or pelvic fins to look for and confirm the discrimination between these species. Similar results were obtained (Figure~\ref{fig:famd12}).

The third axis (explaining 12.2\% of the total variance) mainly carries traits linked to the foraging and swimming behavior, with a strong influence of \emph{anus position} ($a = 23.8$), \emph{gill raker type} ($a = 19.2$), \emph{dorsal fin insertion} ($a = 15.5$) and \emph{pectoral fin position} ($a = 14.2$) (Table~\ref{table:cont_abs}). On the left of this horizontal axis, species are characterized by a rather high pectoral fin insertion and close-to-head dorsal fin insertion and anus position (Figures~\ref{fig:famd34} and \ref{fig:corr_circ_34}).
Finally, the fourth axis (7.9\% of the total variance) is mainly related to feeding and habitat traits, with \emph{pyloric caeca} ($a = 31.6$), \emph{operculum volume} ($a = 17$) and \emph{presence of photophores} ($a = 10.9$) (Table~\ref{table:cont_abs}). Because \textit{X. copei} is the only species having pyloric caeca, this variable clearly separates that species from the others, on the PC4 axis. The remaining PC scores (20\% of variance) are mainly linked to food acquisition. Overall, this FAMD analysis detects similarities between species niches', with some ellipses largely overlapping. Conversely, some species are well segregated along these first four axes, suggesting the existence of several functional groups within studied community. 

PC1 and PC2 carry most of the information in the functional analysis. This suggests that the traits strongly associated with these axes are the most relevant to segregate species. In particular, \emph{lower jaw length}, \emph{body depth}, \emph{pectoral fin insertion} and \emph{eye size} are the quantitative traits that are the most distinct among species and helps to distinguish them. As for qualitative variables, it seems that \emph{oral gape axis} and \emph{gill raker type} are the ones separating the most species. This makes sense considering they have the most modalities, allowing a more precise description than binary \emph{presence of photophores} and \emph{pyloric caeca} scores. 


% Table of contributions of variables for 5 first axis
\begin{table}[!htbp]
\centering
\caption[Contribution values of functional traits to the first five principal components]{Contributions of functional traits to the first 5 principal components. In bold, contributions higher than the threshold of $t =4.8$.}
\label{table:cont_abs}
\begin{adjustbox}{max width=1.1\textwidth,center}
\begin{tabular}{>{\bfseries}crrrrrr}
  \toprule
  &  & PC1 & PC2 & PC3 & PC4 & PC5 \\ 
 Function & Trait & (30.5\%) & (21.6\%) & (12.2\%) & (7.9\%) & (7.8\%) \\ 
  \midrule
  \multirow{13}{*}{Feeding} &Oral gape axis & \textbf{11.07} & \textbf{12.28} & 3.32 & 3.27 & \textbf{7.13} \\ 
  &Eye size & 1.29 & \textbf{9.01} & 3.52 & 4.08 & 1.40 \\ 
  &Orbital length & \textbf{8.35} & 2.38 & 0.00 & 2.24 & 4.54 \\ 
  &Oral gape surface & \textbf{7.78} & 3.06 & 2.93 & 0.10 & 2.78 \\ 
  &Oral gape shape & 0.18 & 3.10 & 0.05 & \textbf{9.49} & \textbf{10.41} \\ 
  &Oral gape position & 0.42 & 1.21 & 1.99 & \textbf{9.16} & 3.94 \\ 
  &Lower jaw length & \textbf{11.52} & 0.00 & 0.04 & 1.97 & 2.44 \\ 
  &Gill raker type & 4.38 & \textbf{17.80} & \textbf{19.23} & 0.32 & 3.42 \\ 
  &Gill outflow & \textbf{5.47} & 0.82 & 0.02 & 2.22 & \textbf{22.48} \\ 
  &Head length & \textbf{8.21} & 3.10 & 2.50 & 0.78 & 4.50 \\ 
  &Pyloric caeca & 0.23 & 0.48 & 0.04 & \textbf{31.62} & \textbf{5.81} \\ 
  &Anus position & 0.00 & \textbf{5.28} & \textbf{23.81} & 0.07 & 0.77 \\ 
   \midrule
   \multirow{6}{*}{Locomotion} & Body depth & \textbf{11.68} & 0.93 & 1.41 & 0.64 & 1.92 \\ 
  &Pectoral fin position & 1.72 & \textbf{7.15} & \textbf{14.22} & 0.37 & 0.32 \\ 
  &Pectoral fin insertion & \textbf{4.95} & \textbf{11.39} & 0.05 & 0.12 & 0.74 \\ 
  &Transversal shape & \textbf{5.96} & 4.33 & 5.45 & 1.70 & 2.10 \\ 
  &Caudal throttle width & 3.72 & \textbf{5.01} & 0.28 & 1.60 & \textbf{16.30} \\ 
  &Dorsal fin insertion & 2.58 & 4.78 & \textbf{15.52} & 0.00 & 1.33 \\ 
  \midrule
  \multirow{2}{*}{Habitat} &Eye position & 0.19 & \textbf{5.41} & 0.78 & 2.34 & 1.40 \\ 
  &Presence photophores & \textbf{7.79} & 2.45 & 3.23 & \textbf{10.89} & 0.59 \\
  &Operculum volume & 2.52 & 0.03 & 1.61 & \textbf{17.02} & \textbf{5.68} \\ 
   \bottomrule
\end{tabular}
\end{adjustbox}
\end{table}

% Plot of FAMD axis 1 and 2
\newgeometry{left=1cm,right=1cm,bottom=1.5cm,top=1cm}
\begin{figure} [!htbp]
	\begin{center}
		\includegraphics[width=\textwidth]{FAMD12.png}  
	\end{center}
	\caption[FAMD results for first and second axis]{FAMD results for PC 1 and 2. Labels correspond to the levels of 4 catagorical variables: \emph{presence of pyloric caeca} (0 or 1), \emph{gill raker type} (A, B or C), \emph{presence of photophores} (present or absent) and \emph{oral gape axis} (superior, terminal or supra-terminal).}
	\label{fig:famd12}
\end{figure}

% Plot of FAMD axis 3 and 4
\begin{figure} [!htbp]
	\begin{center}
		\includegraphics[width=\textwidth]{FAMD34.png}
	\end{center}
	\caption[FAMD results for third and fourth axis]{FAMD results for PC 3 and 4. Labels correspond to the levels of 4 catagorical variables: \emph{presence of pyloric caeca} (0 or 1), \emph{gill raker type} (A, B or C), \emph{presence of photophores} (present or absent) and \emph{oral gape axis} (superior, terminal or supra-terminal).}
	\label{fig:famd34}
\end{figure}
\restoregeometry


\subsection{Functional niche surface and overlap}
Overall, several species' ellipses overlap on these first four principal components (Figures~\ref{fig:famd12} and \ref{fig:famd34}). For these species, the aim is now to measure the extent of these overlaps. On the first and second principal components, all 4 Myctophidae and 2 Platytroctidae species, plus \textit{X. copei} are more or less overlapping, not displaying much differences along PC1 nor PC2 (Figure~\ref{fig:famd12}). \textit{A. risso} and \textit{S. boa} seems to slightly overlap each other niches, which are mainly separated along PC2 (Figure~\ref{fig:famd12}). \textit{S. beanii} and \textit{A. olfersii} are the only two species not showing any overlap. 
On the third and fourth principal components, \textit{A. risso} and \textit{A. olfersii} are showing a partial overlap (Figure~\ref{fig:famd34}). Once again, all 4 Myctophidae and 2 Platytroctidae species overlap. Especially, \textit{N. operosus} and \textit{S. koefoedi} have very similar looking niches which are almost fully overlapping. Among this group of species, \textit{L. crocodilus} has the widest niche. It overlaps widely with \textit{C. maderensis}, \textit{M. punctatum} and \textit{N. kroyeri}'s niches. 

The niche surface estimation showed that \textit{N. kroyeri} had the narrowest niche among the studied community, followed by \textit{N. operosus} and \textit{S. koeofedi} (Table~\ref{table:sp_surface}). Then, \textit{C. maderensis}, \textit{A. risso}, \textit{S. beanii} and \textit{M. punctatum}, \textit{S. boa} and \textit{X. copei} all have niches almost two to three times wider than \textit{N. kroyeri}'s. Finally, compared to \textit{N. kroyeri}'s niche, \textit{L. crocodilus}'s is four times wider, while \textit{A. olfersii}'s is more than seven times wider. In fact, all Myctophidae (except \textit{L. crocodilus}) and Platytroctidae have rather small niches and are the ones overlapping the most in Figure \ref{fig:famd12}.

The effect of sample size on the estimate of the average niche area, calculated by bootstrap, showed no significant variation with \textit{n} (Figure~\ref{fig:niche_est}. Hence, the area of the niches seems to be influenced more by the variability of the traits of the individuals sampled than the number of inviduals itself. Yet, the estimated values of the area of the niches converge quickly for $n >= 30$. 

% Niche standardardised surfaces
\begin{table}[!htbp]
\centering
\caption[Standardized surface of functional niches]{Standardized surface of functional niches.}
\label{table:sp_surface}
\begin{tabular}{lr}
  \toprule
Species & Relative surface \\ 
  \midrule
  \emph{Argyropelecus olfersii} & 7.4 \\ 
  \emph{Lampanyctus crocodilus} & 4.0 \\ 
  \emph{Xenodermichthys copei} & 2.6 \\ 
  \emph{Stomias boa} & 2.5 \\ 
  \emph{Myctophum punctatum} & 2.1 \\ 
  \emph{Serrivomer beanii} & 1.9 \\ 
  \emph{Arctozenus risso} & 1.8 \\ 
  \emph{Ceratoscopelus maderensis} & 1.7 \\ 
  \emph{Searsia koefoedi} & 1.3 \\ 
  \emph{Normichthys operosus} & 1.2 \\ 
  \emph{Notoscopelus kroyeri} & 1.0 \\ 
   \bottomrule
\end{tabular}
\end{table}

Nine species show overlapping niches with at least one other species (Figure~\ref{fig:famd12}) and the smallest niches are the ones overlapping each other the most (Tables~\ref{table:sp_surface} and \ref{table:ell_ovlp}). The maximum  overlap is found between \textit{M. punctatum} and \textit{N. kroyeri}, as almost 69\% of the latter's niche is covered by the first. In total, these two species share 22\% of their niches. \textit{S. koefoedi} is sharing more than 28\% of its niche with both \textit{C. maderensis} and \textit{N. kroyeri}, which are the two highest total overlap values here (Table \ref{table:ell_ovlp}). Overall, \textit{N. kroyeri}'s niche is fully overlapped by six other species. \textit{C. maderensis}, \textit{S. koefoedi} and \textit{M. punctatum} are overlapping five other species. Finally, \textit{L. crocodilus}, \textit{X. copei} and \textit{N. operosus} are involved in overlap with three other species, and \textit{A. risso} and \textit{S. boa} are only overlapping each other (Table \ref{table:ell_ovlp}). Finally, \textit{N. kroyeri} and \textit{N. operosus} show minimum total overlap value of 0.2\%, which might no be significative and mostly due to some individuals of the latter in \textit{N. kroyeri}'s ellipse (Figure \ref{fig:famd12}). 

\begin{table}[!htbp]
\centering
\caption[Overlap of species' ellipses]{Species' ellipses overlap. Here, we present only pairs of species for which an overlap greater than 0 exists.}
\label{table:ell_ovlp}
\begin{tabular}{llrrr}
  \toprule
 &                    & Total        &  Overlap of Species 2 & Overlap of Species 1 \\ 
Species 1 & Species 2 & overlap (\%) &  over Species 1 (\%)  & over Species 2 (\%) \\ 
  \midrule
  \emph{S. koefoedi} & \emph{C. maderensis} & 28.2 & 65.4 & 49.6 \\ 
  \emph{N. kroyeri} & \emph{S. koefoedi} & 28.1 & 65.4 & 49.2 \\ 
  \emph{M. punctatum} & \emph{N. kroyeri} & 22.0 & 32.3 & 68.7 \\ 
  \emph{M. punctatum} & \emph{X. copei} & 20.2 & 44.9 & 36.6 \\ 
  \emph{C. maderensis} & \emph{N. operosus} & 19.0 & 31.8 & 47.1 \\ 
  \emph{N. kroyeri} & \emph{C. maderensis} & 15.1 & 41.6 & 23.7 \\ 
  \emph{M. punctatum} & \emph{S. koefoedi} & 12.5 & 20.3 & 32.4 \\ 
  \emph{S. koefoedi} & \emph{N. operosus} & 7.9 & 15.0 & 16.9 \\ 
  \emph{N. kroyeri} & \emph{X. copei} & 6.4 & 22.9 & 8.8 \\ 
  \emph{S. koefoedi} & \emph{X. copei} & 5.9 & 17.5 & 8.9 \\ 
  \emph{L. crocodilus} & \emph{M. punctatum} & 4.2 & 6.5 & 12.2 \\ 
  \emph{A. risso} & \emph{S. boa} & 3.9 & 9.4 & 6.6 \\ 
  \emph{M. punctatum} & \emph{C. maderensis} & 3.2 & 5.8 & 7.0 \\ 
  \emph{L. crocodilus} & \emph{C. maderensis} & 1.7 & 2.5 & 5.7 \\ 
  \emph{L. crocodilus} & \emph{N. kroyeri} & 1.0 & 1.2 & 4.8 \\ 
  \emph{N. kroyeri} & \emph{N. operosus} & 0.2 & 0.5 & 0.4 \\ 
   \bottomrule
\end{tabular}
\end{table}

Niches distinctiveness informs on functional diversity of species within a community, with increasing values meaning that, locally, species display rarer functions. Here, Myctophidae and Playtroctidae families are all equally functionnally distinct from one another (Table \ref{table:nich_diss}). Even if the values are rather close, \textit{S. koefoedi} and \textit{M. punctatum} seems to be the most alike species. Conversely, \textit{A. olfersii}, \textit{S. beanii} and \textit{S. boa} tend to have more distinct niches, which is consistent with previous results (Figure~\ref{fig:famd12}). Here, species with the most atypical morphological feature are the most distinct in terms of niche. 

\begin{table}[!htbp]
\centering
\caption[Dissimilarity values of species' niches]{Niche dissimilarity of studied species}
\label{table:nich_diss}
\begin{adjustbox}{max width=1.1\textwidth,center}
\begin{tabular}{lr}
  \toprule
Species & Distinctiveness value \\ 
  \midrule
  \emph{Argyropelecus olfersii} & 0.54 \\ 
  \emph{Serrivomer beanii} & 0.51 \\ 
  \emph{Stomias boa} & 0.46 \\ 
  \emph{Arctozenus risso} & 0.40 \\ 
  \emph{Lampanyctus crocodilus} & 0.37 \\ 
  \emph{Xenodermichthys copei} & 0.37 \\ 
  \emph{Ceratoscopelus maderensis} & 0.31 \\ 
  \emph{Normichthys operosus} & 0.31 \\ 
  \emph{Notoscopelus kroyeri} & 0.30 \\ 
  \emph{Searsia koefoedi} & 0.29 \\ 
  \emph{Myctophum punctatum} & 0.29 \\  
   \bottomrule
\end{tabular}
\end{adjustbox}
\end{table}

% \textbf{Si tu veux gagner de la place, tu dois pouvoir fusionner les tables 4 et 6}


\subsection{Kernel density estimation}
Kernel density estimation helps to understand which the traits have the same distribution for several species. Looking at these distributions for species with overlapping niches gives informations on traits, and thus, functions similarities among species. The estimation of kernel densities for the 7 overlapping species show that these species are overlapping for 7 of the 17 computed traits (Figure~\ref{fig:dpo}; Table~\ref{table:kern_over_val}). The overlap is maximum for \emph{oral gape position} (trait n°5) with an overlap value of 0.34. This means that, along this functional trait, these 6 species share nearly 34\% of the trait density distribution. For this trait, most of the distributions are centered around of values of [0.5-0.8], yet \textit{L. crocodilus} displays a pretty wide distribution, despite being the most sampled species (Table~\ref{table:spcount}). The same observation can be done for this particular species for \emph{body depth} (trait n°11) and \emph{pectoral fin insertion} (trait n°13). Species share also nearly 25\% and 29\% of their density when looking at \emph{body depth} (trait n°11) and\emph{eye position} (trait n°17) respectively. Finally, \emph{lower jaw length} (trait n°6) and \emph{pectoral fin position} (trait n°12) values also seem to be common to those species, with 16\% and 19\% of density shared, respectively. To a lesser extent, species share nearly 12\% of the \emph{eye size} values (trait n°2).

Furthermore, we can notice that \emph{oral gape surface} (trait n°3), \emph{operculum volume} (trait n°8) and \emph{pectoral fin insertion} (trait n°13) display bimodal distributions (Figure~\ref{fig:dpo}. For these traits, \textit{L. crocodilus}, \textit{N. operosus} and \textit{S. koefoedi} constitutes the first mode, and the other species the second. Conversely, \emph{eye size} (trait n°1) shows multimodal distribution, with nearly every species having its own distinct mode.

For most traits, \textit{N. kroyeri} (green) and \textit{M. punctatum} (blue) have very similar distributions (Figure~\ref{fig:dpo}). When looking at these two species only, the analysis shows that they are overlapping for each of the 17 traits, with particularly high values for \emph{eye size} (overlap $o = 0.915$), \emph{operculum volume} ($o = 0.901$), \emph{gill outflow} ($o = 0.882$), \emph{pectoral fin insertion} ($o = 0.828$) and \emph{caudal throttle width} ($o = 0.782$), with the mean overlap of the 17 traits being $\bar{o} = 0.561$.


\begin{table}[!htbp]
\centering
\caption[Kernel density overlap values for the 17 traits]{Kernel density overlap values for the 17 computed traits for the seven overlapping species. Values in bold are significant ($p < 0.01$) and shown in Figure~\ref{fig:dpo}.}
\label{table:kern_over_val}
\begin{tabular}{llr}
  \toprule
Trait code & Functional trait & Total overlap (o) \\ 
  \midrule
1 & Eye size & 0.00 \\ 
  2 & Orbital length & \textbf{0.12} \\ 
  3 & Oral gape surface & 0.00 \\ 
  4 & Oral gape shape &0.00 \\ 
  5 & Oral gape position & \textbf{0.34} \\ 
  6 & Lower jaw length & \textbf{0.16} \\ 
  7 & Gill outflow & 0.00 \\ 
  8 & Operculum volume & 0.00 \\ 
  9 & Head length & 0.00 \\ 
  10 & Anus position & 0.00 \\ 
  11 & Body depth & \textbf{0.25} \\ 
  12 & Pectoral fin position & \textbf{0.20} \\ 
  13 & Pectoral fin insertion & \textbf{0.01} \\ 
  14 & Transversal shape & 0.00 \\ 
  15 & Caudal throttle width & 0.00 \\ 
  16 & Dorsal fin insertion & 0.00 \\ 
  17 & Eye position & \textbf{0.29} \\ 
   \bottomrule
\end{tabular}
\end{table} 

\begin{figure} [!htbp]
	\begin{center}
		\includegraphics[width=\textwidth]{Density_plot.png}
	\end{center}
	\caption[Kernel density distribution of overlapping species]{Kernel density overlap for 11 functional traits and the 7 overlaping species. Colors correspond to following species: orange - \textit{L. crocodilus}; brown - \textit{C. maderensis}; yellow - \textit{N. operosus}; purple - \textit{X. copei}; green - \textit{N. kroyeri}; blue - \textit{M. punctatum}; grey - \textit{S. koefoedi}. Overlap is represented by shaded lines. Here, are displayed every traits that showed non-null overlap. Codes for traits are the same as in Table~\ref{table:kern_over_val}.}
	\label{fig:dpo}
\end{figure}