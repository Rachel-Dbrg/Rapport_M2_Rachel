%!TEX root = ./Structure_rapport_final.tex


Because studies of functional diversity on submarine canyons communities are scarce, comparisons with other systems are limited. This work provides a first overview of how deep-sea fish functional niches are structured in the canyons of the Bay of Biscay. We found that deep-sea fish species displayed significant functional diversity, which is consistent with \citet{aneeshkumar2017} and \citet{carrington2021}. Some of these species being functionally specialized, suggesting low competitive interactions, conversely to what might be expected in such a nutrient-poor environment. These results are in agreement with \citet{preciado2017}, who showed that these communities do not seem to display high dietary overlap, indicating a high degree of resources partitioning. Finally, overlap even seems to decrease with depth, because specialization is stronger \citep{carrasson2002,farre2016}. 

Furthermore, having better knowledge of these communities is crucial to evaluate their vulnerability facing disturbance of ecosystem. \citet{carrington2021} highlighted that, if diversity in deep-sea ecosystems is not to be questioned, very little is known about the way species inhabiting these systems might respond to pressures shifting. Because diversity range within an ecosystem is a good indicator of its health, measuring functional diversity might help quantify the effect of environmental disturbance \citet{carrington2021,villeger2017}. Despite being, currently, rather spared by anthropic influence, deep-sea ecosystems are nonetheless facing several threats. With the improvement of technology, deep-sea exploitation, and especially fishing, is increasing \citep{carrington2021}. Yet, because of their slow life histories and the complexity of their diversity, deep-sea communities might be more sensitive to biodiversity loss than shallow-water communities \citep{carrington2021,danovaro2017}. Another concerning issue that deep-sea communities is facing is ocean pollution. In particular, submarine canyons being conduits linking shallow to deeper waters, they are particularly exposed to pollution coming from continental shelf. In particular, studies have shown high density of marine litter in Bay of Biscay's canyons \citep{vandenbeld2017}. Being mainly microplastic, this pollution affects all water column, from pelagic to deep-sea species \citep{pereira2020}. Metallic and organic pollutants contamination have also been reported for deep-sea communities \citep{spitz2019}.

Finally, climate change could be the most influent yet harder to quantify threat of all for these communities. Inhabiting deep-sea environments for millions of years, meso- and baythpelagic species had to adapt to previous changes, but none was as fast and intense as the current one \citep{catul2011}. In particular, by 2100, ocean warming is projected to induce a 1°C increase in deep-waters, reducing oxygen concentrations and acidifying water \citep{danovaro2017}. These changes are foreseen to be even more important in shallower waters, on which deep-sea communities feed on, with already happening changes \citep{danovaro2017}. Ultimately, this will affect capacity of these community to recover from other disturbance, with potential detrimental impacts on their metabolism, growth rate and reproduction success. Because of their essential role, ensuring energy transfer from shallow to deep-waters, health of meso- and bathypelagic communities is key determining the whole trophic chain \citep{davison2015,gaskett2001}. If deep-sea fish abundance were to decrease, the established relations between species in competition will change, with repercussion on higher trophic levels. \citet{aissi2012} and \citet{kenchington2020} emphasized the importance of canyons deep-sea communities as major food supply supporting whale, large pelagic predators and also marine birds. 

To mitigate global change effects on deep-sea communities, and thus others compartments, an effective management of these resources have to be decided. To do so, a much complete understanding of functional diversity and composition of these communities is required, and could help predict impact and responses to global change and exploitation \citep{carrington2021,dumay2004,kremer2017}. Special effort should be made to expand knowledge on biology, life history and trophic relations of deep-sea communities, which is essential to address the additive effects of human pressures and global change \citep{danovaro2017}.


% CCL :
% - ici 4 top prédators qui ont des niches très différentes
% - miallons trphiques infs ont des niches qui se chevauchent + d