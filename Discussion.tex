%!TEX root = ./Structure_rapport_final.tex

The results of our study provide evidence of both segregation and overlap of functional niches within the deep-sea fish community inhabiting canyons of the Bay of Biscay. Specifically, differences among species were primarly shaped by traits referring to body and head morphology and by foraging related traits, which also give information on their habitat \citep{schoenfuss2007}. Most of the detected overlap is found between some Myctophidae, Platytroctidae and Alepocephalidae species. However, the overlap appeared to be higher between families rather thant within families. Species with the most atypical morphology (\emph{i.e} \textit{A. risso}, \textit{A. olfersii}, \textit{S. beanii} and \textit{S. boa}) have clearly distinct niches and showed null/low overlap, within the community. These results reveals that the most common species of deep-sea communities of the Bay of Biscay show a relatively high functional diversity and several degree of segregation. This supports the hypothesis of a high partitioning of resources in these ecosystems and promotes coexistence of species, which might be essentially driven by food resources, but also swimming abilities \citep{aneeshkumar2017,preciado2017}.\\

% \subsection{Limits and biais}

No sex or maturity distinction was made when measuring individuals, and this source of variability was not investigated in our analysis. For fishes, both variables could influence morphologic features: sexual dimorphism being mostly related to standard length and fins position and maturity to prey selection \citep{geidner2008,nagelkerke2018}. Such difference could explain a part of trait variability and lead to very different results of niche partitionning \citep{dasilva2019,dumay2004,nagelkerke2018}.

In this study, the relationship between species functional traits and environmental (biotic or abiotic) parameters were not investigated. In an ecosystem, species interact also with their environment. Whether it shapes the habitat or define availability of resource breadth, environment is influencing realized niche \citep{costa-pereira2019,ibanez2007,kremer2017}. Adding these parameters in future studies could potentially allow to associate the different niches to particular environmental characteristics, as well as investigate site influence. Because traits are morphological responses to environmental constraints, their interpretation could vary from one context to another, making it hard to extract the main drivers of trait variation. \citep{kremer2017}. Comparing communities living in different sites is relevant to link some functional traits with habitat particularities, thus assess how environment influence functional niches \citep{mejri2009}. Moreover, including local environmental conditions could explain the part of trait variability due to pressures from ecological divergence \citep{dasilva2019}. 

In a context of climate change, studying this relationship is important to predict how disturbances of the habitat could affect fish communities \citep{brindamour2011}. Finally, previous studies showed that functional niche overlap and species abundances were linked \citep{aneeshkumar2017,mason2008}. More precisely, the most abundant species of an ecosystem tend to be functionnally separated from the others \citep{farre2016}. For future studies, species' abundances could be taken into account, and could help refine functional diversity estimations. 


% Similarly, potential seasonal effect on functional traits was not assessed. Because plankton diversity and abundance is varying with seasons, species that directly relies on this resource as their main food resource might express seasonal patterns in their food-related traits \citep{kremer2017}.

% For Bay of Biscay's canyons specifically, no studies assessing the species diversity and abundance were found. Looking at literature from other studies meso- and bathypelagic communities give divergent results. For \citet{garcia2021} and \cite{kenchington2020}, deep-sea pelagic fish assemblages are dominated by Gonostomatidae (famility not studied here), \citet{eduardo2020} considers Sternoptychidae (family of \textit{A. olfersii}) as one of the most conspicuous components of the mesopelagic ichthyofauna and \citep{sutton2008} refers to \textit{S. boa} and \textit{S. beanii} as the most abundant species. Therefore, explicit work of diversity and abundance of fishes inhabiting Bay of Biscay canyons might be needed to have more precise functional niche overlap estimations.

Here, morphological measurements alone highlighted clear differences among species, with segregation along functional traits. However, using only morphological traits to describe functional niche segregation has some limits \citep{dasilva2019}. \citet{grossman2009} showed that species with strong morphological divergences sometimes use the same resources in a similar way, whereas morphologically close species can have very different diets. In order to overcome the limits imposed by their morphology and/or habitat for the acquisition of food, organisms can indeed modulate their behavior and show great adaptability \citep{blondel1979,grossman2009}. Altough morphology sets limits to resource use, species can exhibit some plasticity to adapt to prey availability and environmental conditions \citep{ibanez2007,sibbing2000}. Completing present results with data on dietary preferences could increase robustness and confirm first conclusions. To do so, \citet{preciado2017} suggests that both stable isotope analysis and stomach content analysis give complementary, or even opposites information. Considering few information known about present studied species, additional analysis might also help drawing conclusions from functional analysis. Especially, examining stomach content can be useful when not enough functional traits were computed, or if the latter can not explain much of the variability found between species \citep{albouy2011}. This will also inform on how, for deep-sea species communities, functional traits change regarding trophic level or diet, and thus, help interpretation of functional diversity assessment. \\

% \subsection{Community structure in Bay of Biscay's canyons}

% For deep-sea fish species, resource partionning is mainly linked to the size of prey and swimming ability \citep{aneeshkumar2017}. Our results showed clear dissimilarities in functional traits, related to habitat, feeding and locomotion meaning that studied species display differences in the way they use resources. Especially, species seems to be highly separated by the way they feed, which also give informations on their habitat \citep{schoenfuss2007}. Indeed, based on our results, body and head morphology as well as food prospection and acquisition separates species in four distinct groups.


% RAJOUTER PARTIE SUR l'EXPLICATION DES AXES ?

% PC4 --> linked to prey selection

First, \textit{A. olfersii} have the most distinct and widest niche of all studied species with very differentiated morphological and habitat-related traits. \textit{A. olfersii} showed an atypical laterally compressed body, long lower jaw, superior-oriented gape-axis and absent/rudimentary gill rakers. All these elements are consistent with \textit{A. olfersii} being a predator, mainly feeding on teleostei juveniles, euphausiids and gelatinous organisms \citep{eduardo2020}. These morphological features suggests that this species catch its prey from underneath, with what might be a suction mechanism. On the opposite side of the morphologic spectrum are found the elongated, short-head fishes, with (supra-) terminal-oriented mouth. 
% Because running the same analysis without pectoral fin related traits did not lead to \textit{S. beanii} overlapping neither \textit{A. risso} or \textit{S. boa}' ellipse, the differences observed between these three species mainly relies on gill rakers type (absent for \textit{S. beanii}, low-developped for \textit{A. risso} and \textit{S. boa}) and eye size, which are bigger for the two latest.

All these three species being primarily piscivorous top predators with sharp teeth, \textit{S. beanii} has been reported having a generalist diet, feeding on wide range of preys (cephalopods, teleosts, crustaceans \ldots{}) \citep{geidner2008}. For generalist species, \citet{sibbing2000} emphasize the difficulty to link morphology to foraging characteristics and/or diet, because such species are able to switch prey depending on which is more abundant, and do not exhibit specific food-related morphological traits. Ultimately, such mixed diet can require less time and energy prospecting for food \citep{geidner2008}. This is consistent with \textit{S. beanii} which seems to have a different swimming strategy than \textit{A. risso} and \textit{S. boa} (see PC3 on Figure \ref{fig:famd34}). Conversely, the latter has a rather selective diet which is dependent of their foraging strategy \citep{sutton1996}. Stomiids use lures to attract prey, they are dependent on prey appeal for lures to feed and target rather big ones, which narrows the range of potential preys \citep{geidner2008,germain2019}. 
% Yet, \textit{S. boa} seems to have a rather spreaded niche surface (see \ref{table:sp_surface}), which might be explained by the fact it mainly feeds on Myctophidae, which is the most diversed family of the meso- bathypelagic ecosystems \citep{garcia2021,sutton1996}. 

For the moment, being the only known species of its genus, very little is known about \textit{A. risso} and recent studies highlighted confusions between this particular species and what might be a new species, based on morphological differences \citep{ho2019}. Looking at functional diversity through morphology can thus play a role in assessing local biodiversity and help distinguish species that look alike. \citet{vesk2021} recent work emphasize that species traits give information to understand the why and where species occured in an environment, and the parameters that makes this environment suitable. Looking at species traits can thus be a promessing tool to predict environmental conditions and species distribution.  


% Though lack of literature limits their intrepretation, our results suggests that both \textit{A. olfersii} and \textit{A.risso} might feed on same preys. If the first tend to have a wider spreaded global niche (see Table \ref{table:sp_surface}), the latter seems to have a wider range of potential preys (see Figure \ref{fig:famd34}). Moreover, \textit{A. olfersii} is the only species that has a stretched niche along PC3, while all other species displayed on Figure \ref{fig:famd34} tend to have it stretched along PC4 . This might indicate a narrower range of prey diversity, which could be linked to the differences in feeding behavior showed on PC1 (see Figure \ref{fig:famd12}) . 
 

Overall, \textit{S. beanii}, \textit{A. risso}, \textit{A. olfersii} and \textit{S. boa} all have the most distinct niches among all eleven studied species and live in deeper waters \citep{froese2019}. These species all are mesopelagic top predators, confirming the observed relation between oral anatomy and trophic level \citep{colborne2013,wainwright1995}. Indeed, trophic level impacts how specialised a species can be in terms of diet: apex predators will often favor one feeding strategies among others, so they are very efficient for one strategy, and have limitating capacities in others. This principle can be summed up as a ``trade-off strategy'': greater abilities for one strategy lead to a decrease of abilities in other areas \citet{norton1995}, because of morphological specificities \citep{nagelkerke2018}. Consistent with \citet{farre2016} observations, redundancy seems to decrease with depth, as these four deep-living species show null/low redundancy. Therefore, these four species occupy different niches and should have low competitive interactions \citep{mouillot2005}. 

On the other hand, higher overlap among species belonging to Myctophidae and Platytroctidae . Indeed, the six species have close niches based on the morphology of their body and head (PC1), their feeding strategies (PC2), their diet (PC3) and prey selection (PC4), suggesting similar use of resources. Morphologically, these species are all characterized by a rather large and elongated body, supra-terminal oral orientation, well-developed gill rakers and the presence of photophores (except \textit{N. operosus}). Both families have relatively big eyes, which are highly specialized for dim-light environments and detecting/localizing sources of light \citep{debusserolles2014,novotny2018}. Platytroctidae also display a luminous organ, called shoulder organ, that produces light through bioluminescent fluid that appears to disorient predators \citep{novotny2018}. However, Myctophidae perform diel vertical migration to feed on wide variety of shallower-water zooplankton preys, which Platyctroctidae do not \citep{sutton2013a}. Furthermore, Myctophidae are reported to be opportunistic predators, feeding on euphausiids, copepods, ostracods, fish eggs and larvae, depending on the available food and the habitat \citet{catul2011,kozlov1995}. Yet, Platytroctidae seems to have a more specialized diet, not being truly a generalist \citet{novotny2018}. 

Because all six species have rather close niches, size of niche might therefore inform on the way species partition resources. One hypothesis might be that species with wider niche have higher flexibility in resource utilization and can thus feed on more diverse preys. For example, \textit{L. crocodilus} wide feeding-niche can be explained by seasonal variations of preferred preys type \citet{fanelli2014}. Conversely, species that have small niches might have more restricted diet variability, because it is equally abundant all year or because they are not morphologically able to feed on other preys \citep{mejri2009}. Another hypothesis is the competition theory in which communities' structures are constraints by competition relationships, which make sense here regarding the similarity in niches of these species \citep{geange2011}. Under this hypothesis, the size of the niche is directly influenced by the availability of prey, which depends on what the other species of the community predate and how resources are shared.

In our case, \textit{N. kroyeri} and \textit{M. punctatum} have highly overlapped niches, the latter overlapping nearly 69\% of the first. Because \textit{M. punctatum}'s niche is twice wider, this might suggest that while both species are competiting for the same resource, \textit{N. kroyeri} is limited in its resource use by \textit{M. punctatum}. For these two species, phylogenic closeness might explain the likeness of niches, as it is linked to functional-trait diversity \citep{tucker2018}. Indeed, traits density estimation showed very similar results for both species, which are 56\% overlapping on the 17 functional-traits studied. Overlap is the highest for traits that are linked foraging strategies, suggesting that while the two species use the same resource, they also do it in the same way. Hence, \textit{N. kroyeri} and \textit{M. punctatum} can theoretically compete, especially if their food is limiting.

Finally, \textit{X. copei}, which is not Myctophidae nor Platyctroctidae but Alepocephalidae, show only a slight overlapping three with species from the first two families. Yet, \textit{X. copei} has the third widest realized niche of all studied species, mostly elongated, indicating a rather high morphological variability (see Table \ref{table:sp_surface} \& PC1 on Figure \ref{fig:famd12}). This species being the only one here to have pyloric caeca, its relative segregation might be influenced by this specific feature, which plays a role in digestion abilities \citep{buddington1986}. Another explanation might be the high variability of prey they feed on, depending on the habitat they are living in. \citet{mauchline1983} indeed showed that, for this species, diet varied along life stage, with juveniles feeding exclusively on copepods and ostracods, while adults showed a much more diverse diet, including euphausiids and other fish from deeper waters. 

Looking at overlap for each traits helps us to understand how \textit{X. copei} resembles to Myctophidae and Platyctroctidae. \textit{X. copei} showed similar trait-density variation with most species of these two families, being the closest with \textit{M. punctatum}. In particular, these two species are similar regarding at morphological features, such as body depth, pectoral fin position and insertion. For this latter trait, high resemblance is found between \textit{X. copei} and Myctophidae (except \textit{L. crocodilus}), while the two Platyctroctidae species look similar to one another. Finally, considering all these traits together suggest that despite high morphological and locomotion similarities, \textit{X. copei} is unlikely to compete with neither Myctophidae or Platyctroctidae, because they are targeting different preys. 

Here, effect of phylogeny is not that obivous. For example, despite Alepocephalidae being evolutionnary close to Platyctroctidae (see Figure \ref{fig:phylotree}), null/low overlap is found. It is also interesting to notice that despite close phylogenic history between \textit{S. boa} and \textit{A. olfersii}, both belonging to Stomiiformes order, no similarities in niche's occupation were found. This suggest that, when looking at how species share resources, functional traits overrides some effects of phylogeny \citep{kremer2017}. However our results tend to show that, within Myctophidae and Platyctroctidae, overlap is lower between species than between families. Being close from a phylogeny perspective, this observation might suggest that selectives forces have led to evolutionnary divergence within each of these families. As it applied to both families, this divergence caused what seems to be greater resemblance between families than inside families. Indeed, species that took different genetic paths (\textit{i.e} evolution of two distinct families example) could end reaching same traits \citep{natarajan2016}. Ultimately, this can lead to greater competition between species from different families, rather than within family, as what could be expected.


% - Comment ça se positionne au sein de ce qui est déjà connu : classique ou pas, 
% breakthrough en comparant avec la littérature. 


% % Methode du boostrap et paramétrisation de l'exp
% Zhao 2014 --> mêmes conclusions sur l'augmentation de la taille de la niche en fct du nbe d'individus.

% communautaire : certaines espèces sont-elles redondantes ? Plusieurs espèces
% occupent la même niche en cas de chevauchement, occupent la même niche fonctionnelle. Se focaliser
% sur les fonctions plutôt que sur l'espèce. 
% 		- Individus appartenant à la meme niche peuvent être considérés comme appartenant
% à la même boîte fonctionelle. 


% Discussion: niches trophiques, comportements, habitats, sensibilité des espèces, 
% particularité des espèces, voir les avantages de partager les niches
