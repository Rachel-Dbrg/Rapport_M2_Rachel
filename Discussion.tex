%!TEX root = ./Structure_rapport_final.tex

\subsection{Community structure in Bay of Biscay's canyons}

For deep-sea fish species, resource partionning is mainly linked to the size of prey and swimming ability \citep{aneeshkumar2017}. Our results showed clear dissimilarities in functional traits, related to habitat, feeding and locomotion meaning that studied species display differences in the way they use resources. Especially, species seems to be highly separated by the way they feed, which also give informations on their habitat \citep{schoenfuss2007}. Indeed, based on our results, body and head morphology as well as food prospection and acquisition separates species in four distinct groups.

First, \textit{A. olfersii} have the most distinct yet biggest niche of all studied species (see Table \ref{table:nich_diss} and Table \ref{table:sp_area}), with very differenciated morphological and habitat-related traits (see Figure \ref{fig:famd12}). \textit{A. olfersii} showed an atypical laterally compressed body, long lower jaw, superior-oriented gape-axis and absent/rudimentary gill rakers (see \ref{fig:git}). All these elements are consistent with \textit{A. olfersii} being a predator, mainly feeding on teleostei juveniles, euphausiids and gelatinous organisms \citep{eduardo2020}. These morphological features suggests that this species catch its preys from underneath, with what might be a succion mecanism. On the opposite side of the morphologic spectrum are found the elongated, short-head fishes, with (supra-) terminal oriented mouth. Because running the same analysis without pectoral fin related traits did not lead to \textit{S. beanii} overlapping neither \textit{A. risso} or \textit{S. boa}' ellipse, the differences observed between these three species on Figure \ref{fig:famd12} mainly relies on gill rakers type (absent for \textit{S. beanii}, low-developped for \textit{A. risso} and \textit{S. boa}) and eye size, which are bigger for the two latest. These characteristics suggests that \textit{A. risso} and \textit{S. boa} have similar functional traits, in terms of morphology and habitat, which explains niches overlap on Figure \ref{fig:famd12}. 

All three species being primarly piscivorous top predators with sharp teeth, \textit{S. beanii} has been reported having a generalist diet, feeding on wide range of preys (cephalopods, teleosts, crustaceans \ldots{}) \citep{geidner2008}. Ultimately, this varied diet require less time and energy prospecting for food \citep{geidner2008}. This is consistent with what can be seen on Figure \ref{fig:famd34}, where \textit{S. beanii}'s seems to have a different swimming strategy (PC3) than \textit{A. risso} and \textit{S. boa}. Conversely, the latter has a rather selective diet which is dependant of their hunting strategie \citep{sutton1996}. Because Stomiids uses lures to attract preys, they are dependants on preys appeal for lures to feed and target rather big ones, which narrows the range of potential preys \citep{geidner2008,germain2019}. Yet, \textit{S. boa} seems to have a rather spreaded niche area (see \ref{table:sp_area}), which might be explained by the fact it mainly feeds on Myctophidae, which is the most diversed family of the meso- bathypelagic ecosystems \citep{garcia2021,sutton1996}. Being opposed in Figure \ref{fig:famd12}, \textit{A.risso} and \textit{A. olfersii} nonetheless seem to share some of their locomotion strategies, with their niche partly overlapping (see Figure \ref{fig:famd34}). For the moment, being the only known species of its genus, very little is known about \textit{A. risso} and recent studies highlights confusions between this particular species and what might be a new species, based on morphological differences \citep{ho2019}. Though lack of literature limits their intrepretation, our results suggests that both \textit{A. olfersii} and \textit{A.risso} might feed on same preys. If the first tend to have a wider spreaded global niche (see Table \ref{table:sp_area}), the latter seems to have a wider range of potential preys (see Figure \ref{fig:famd34}). Finally, it is interesting to notice that despite close phylogenic history between \textit{S. boa} and \textit{A. olfersii}, both belonging to Stomiiformes order (see Figure \ref{fig:phylotree}), no similarities in niche's occupation were found. This suggest that, when looking at how species share resources, functional traits overrides the effects of phylogeny. 

Overall, all four species located in the periphery of Figure \ref{fig:famd12} have the most extremes morphology of all eleven studied species and live in deeper waters \citep{froese2019}. Differenciated from the resting seven species along head anatomy (PC1) and food-acquisition related features (PC2), these species all are mesopelagic top predators, confirming the observed relation between oral anatomy and trophic level \citep{colborne2013,wainwright1995}. Consistent with \citet{farre2016} observations, redundancy seems to decrease with depth, as these four deep-living species show null/low redundancy and have the four most distinct niches (see Table \ref{table:nich_diss}). Therefore, these four species occupies different niches and are not likely to be competiting \citep{mouillot2005}.

On the other hand, high overlap is found in Myctophidae, Platytroctidae and in between these two families (see Figures \ref{fig:famd12} and \ref{fig:famd34}). Indeed, the six species can hardly separated based on the morphology of their body and head (PC1), their feeding strategies (PC2), their diet (PC3) and prey selection (PC4), suggesting similar use of resources. Morphologically, these species are all characterised by a rather large and elongated body, supra-terminal oral orientation, well-developped gill rakers and the presence of photophores (except \textit{N. operosus}). Both families have relatively big eyes, which are highly specialised for dim-light environments and detecting/localizing sources of light \citep{debusserolles2014,novotny2018}. Platytroctidae also display a luminous organ, called shoulder organ, that produces light through bioluminescent fluid that appears to disorient predators \citep{novotny2018}. 

Mainly living at around 1000m deep \citep{froese2019}, Myctophidae perform diel vertical migration to feed on wide variety of shallower-water zooplankton preys, which Platyctroctidae do not \citep{sutton2013a}. Furthermore, Myctophidae are reported to be opportunistic predators, feeding on euphausiids, copepods, ostracods, fish eggs and larvae, depending on the available food and the habitat \citet{catul2011,kozlov1995}. Yet, Platytroctidae seems to have a more specialized diet, not being truly a generalist \citet{novotny2018}. This may explain the slight segregation along PC3, than can be seen on Figure \ref{fig:famd34}, where all Myctophids' niches overlaps one another, while both species of Platytroctidae overlaps each other too. 

When looking at niches areas (see Table \ref{table:sp_area}), \textit{L. crocodilus} indeed displays the widest niche among these two families, four time the size of the smallest one. Eventually, this wider niche lead to overlap with functionally close species, such as \textit{M. punctatum}, \textit{C. maderensis} and \textit{N. kroyeri} (Table \ref{table:ell_ovlp}). The latter conversely display the smallest niche, overlapped by all species of Myctophidae and Platyctroctidae (Table \ref{table:ell_ovlp}). Because all six species have rather similar niches (see Table \ref{table:nich_diss}), size of niche might therefore inform on the way species partition resources. One hypothesis might be that species with wider niche have higher flexibility in resource utilisation and can thus feed on more diversed preys. For example, \textit{L. crocodilus} wide feeding-niche can be explained by seasonal variations of prefered preys \citet{fanelli2014}. Conversely, species that have small niches might no switch prefered preys through time, because it is equally abundant all year or because they are not morphologically equiped to feed on other preys. An other hypothesis is the competition theory in which communities' structures are constraints by competition relationships, which make sense here regarding the similarity in niches of these species \citep{geange2011}. Under this hypothese, the size of the niche is directly influenced by the availability of preys, which depends on what the others species of the community predates and how resources are shared.

In our case, \textit{N. kroyeri} and \textit{M. punctatum} have highly similar niches, the latter overlapping nearly 69\% of the first (Table \ref{table:ell_ovlp}). Because \textit{M. punctatum}'s niche is twice bigger (Table \ref{table:sp_area}), this might suggest that while both species are competiting for the same resource, \textit{N. kroyeri} is limited in its resource use by \textit{M. punctatum}. For these two species, phylogenic closeness might explain the likeness of niches, as it is linked to functional-trait diversity \citep{tucker2018}. Indeed, traits density estimation showed very similar results for both species, which are 56\% overlapping on the 17 functional-traits studied (see Figure \ref{fig:dpo}). Overlap is the highest for traits that are linked to feeding and/or hunting strategies, suggesting that while the two species use the same resource, they also do it in the same way. Hence, \textit{N. kroyeri} and \textit{M. punctatum} could theoretically compete, especially if the preys they feed on were to be less abudant.

Finally, as can be seen on Figure \ref{fig:famd12}, \textit{X. copei}, which is not Myctophidae nor Platyctroctidae, is partly overlapping these latters niches. Similarities are found along PC1 and PC2, which suggests resemblance in morphology and feeding strategy. Thus, no overlapping is found when looking at prey selection (see Figure \ref{fig:famd34}). This species being the only one here to have pyloric caeca, it position along PC4 might be influenced by that specific feature, which plays a role in digestion for fishes. 




PC1 --> Morpho du corps et de la tête
 strongly correlated with shape of the body and of the head with traits such as \emph{body depth} (a = 11.7), \emph{lower jaw length} (a=11.5), \emph{oral gape axis} (a=11.1), orbital length (a = 8.35) and head length (a = 8.21) (Table \ref{table:cont_abs}). Along this axis, all traits have increasing values toward the right (see \ref{fig:corr_circ_12} for details)

PC2 --> habitat (food prospection (loco) et acquisition)
correlated to feeding, through food prospection behaviour with \emph{pectoral fin insertion} (a = 11.4), \emph{eye size} (a = 9), \emph{pectoral fin position} (a = 7.1) and food acquisition through \emph{gill raker type} (a = 17.8) and \emph{oral gape axis} (a = 12,3) (Table \ref{table:cont_abs}).

PC3 --> Diet and swimming strategies
linked to the locomotion, with strong influence of \emph{anus position} (a = 23.8), \emph{gill raker type} (a = 19.2), \emph{dorsal fin insertion} (a = 15.5) and \emph{pectoral fin position} (a = 14.2) (Table \ref{table:cont_abs}). On the left of this axis, species are characterized by a rather high pectoral fin insertion and close-to-head dorsal fin insertion and anus position (see Figures \ref{fig:famd34} and \ref{fig:corr_circ_34}).

PC4 --> Prey selection
 related to traits linked to prey selection, with \emph{pyloric caeca} (a = 31.6), \emph{operculum volume} (a = 17) and \emph{presence of photophores} (a = 10.9) (Table \ref{table:cont_abs}). The remaining PC scores (20\% of variance) are mainly linked to food acquisition. 

ouverture:
 --> Dimorphisme pas pris en compte mais peut jouer sur la morpho --> Geidner 

 --> Previous studies showed that functional niche overlap and species abundances were linked \citep{aneeshkumar2017,mason2008}. More precisely, the most abundant species of an ecosystem tend to be functionnally separated from the others \citep{farre2016}. One explanation is that less-abundant (or rare) species have highly specialized morphological features, which may narrow the range of ressources available to use \citep{aneeshkumar2017}. In this ecosystem, Myctophidae, which dominates in terms of abundance and diversity \citep{catul2011,garcia2021}, seem to share similar functional traits. Moreover, our results showed similarities between Myctophidae and Platytroctidae, the two families overlapping each other niches. --> Vrai pour A. olfersii (voir Eduardo2020) --> l'une des familles les + visibles de ces écsosystèmes
In terms of abundance and biomass, representatives of the family
Sternoptychidae (hatchetfishes) are one of the most conspicuous components
of the mesopelagic ichthyofauna (Gjøsaeter and Kawaguchi,
1980).
Other species showing highest abundance (% of species numbers in parentheses) west of the ridge include: Protomyctophum arcticum (99%), Gonostomatidae (98%, mostly damaged Cyclothone),  Lampanyctus macdonaldi (94%), Holtbyrnia anomala (92%), Serrivomer beanii (84%), Malacosteus niger (82%), and Stomias boa ferox (70%). --> 
\citep{sutton2008}


% % Impact de la taille dr la niche sur les relations trophiques (compet)
% Faire le lien entre la taille de la niche et la compétition : + la niche est petite, plus il y a de proba que la compétition en soit la raison 

% Methode du boostrap et paramétrisation de l'exp
Zhao 2014 --> mêmes conclusions sur l'augmentation de la taille de la niche en fct du nbe d'individus.

communautaire : certaines espèces sont-elles redondantes ? Plusieurs espèces
occupent la même niche en cas de chevauchement, occupent la même niche fonctionnelle. Se focaliser
sur les fonctions plutôt que sur l'espèce. 
		- Individus appartenant à la meme niche peuvent être considérés comme appartenant
à la même boîte fonctionelle. 


Discussion: niches trophiques, comportements, habitats, sensibilité des espèces, 
particularité des espèces, voir les avantages de partager les niches 


Parler des conséquences d'une diminution des proies --> impact sur toute la chaine trophique + compétition entre les espèces avec des niches similaires 