%!TEX root = ./Structure_rapport_final.tex

\subsection{Community structure in Bay of Biscay's canyons}

For deep-sea fish species, resource partionning is mainly linked to the size of prey and swimming ability \citep{aneeshkumar2017}. Our results showed clear dissimilarities in functional traits, related to habitat, feeding and locomotion meaning that studied species display differences in the way they use resources. Especially, species seems to be highly separated by the way they feed, which also give informations on their habitat \citep{schoenfuss2007}. Indeed, based on our results, body and head morphology as well as food prospection and acquisition separates species in four distinct groups. First, \textit{A. olfersii} have the most distinct yet biggest niche of all studied species (see Table \ref{table:nich_diss} and Table \ref{table:sp_area}), with very differenciated morphological and habitat-related traits (see \ref{fig:famd12}). \textit{A. olfersii} showed an atypical laterally compressed body, long lower jaw, superior-oriented gape-axis and absent/rudimentary gill rakers (see \ref{fig:git}). All these elements are consistent with \textit{A. olfersii} being a predator, mainly feeding on teleostei juveniles, euphausiids and gelatinous organisms \citep{eduardo2020}. These morphological features suggests that this species catch its preys from underneath, with what might be a succion mecanism. On the opposite side of the morphologic spectrum are found the elongated, short-head fishes, with (supra-) terminal oriented mouth. Because running the same analysis without pectoral fin related traits did not lead to \textit{S. beanii} overlapping neither \textit{A. risso} or \textit{S. boa}' ellipse, the differences observed between these three species on Figure \ref{fig:famd12} mainly relies on gill rakers type (absent for \textit{S. beanii}, low-developped for \textit{A. risso} and \textit{S. boa}) and eye size, which are bigger for the two latest. These characteristics suggests that \textit{A. risso} and \textit{S. boa} have similar functional traits, in terms of morphology and habitat, which explains niches overlap on Figure \ref{fig:famd12}. All three species being primarly piscivorous top predators with sharp teeth, \textit{S. beanii} has been reported having a generalist diet, feeding on wide range of preys (cephalopods, teleosts, crustaceans \ldots{}) \citep{geidner2008}. Ultimately, this varied diet require less time and energy prospecting for food \citep{geidner2008}. This is consistent with what can be seen on Figure \ref{fig:famd34}, where \textit{S. beanii}'s seems to have a different swimming strategy (PC3) than \textit{A. risso} and \textit{S. boa}. Conversely, the latter has a rather selective diet which is dependant of their hunting strategie \citep{sutton1996}. Because Stomiids uses lures to attract preys, they are dependants on preys appeal for lures to feed and target rather big ones, which narrows the range of potential preys \citep{geidner2008,germain2019}. Yet, \textit{S. boa} seems to have a rather spread niche area (see \ref{table:sp_area}), which might be explained by the fact it mainly feed on Myctophidae, being the most diversed family of the meso- bathypelagic ecosystems \citep{garcia2021,sutton1996}.


one obvious observation is that Myctophidae seem to use resources in very similar ways. 

PC1 --> Morpho du corps et de la tête
 strongly correlated with shape of the body and of the head with traits such as \emph{body depth} (a = 11.7), \emph{lower jaw length} (a=11.5), \emph{oral gape axis} (a=11.1), orbital length (a = 8.35) and head length (a = 8.21) (Table \ref{table:cont_abs}). Along this axis, all traits have increasing values toward the right (see \ref{fig:corr_circ_12} for details)

PC2 --> habitat (food prospection (loco) et acquisition)
correlated to feeding, through food prospection behaviour with \emph{pectoral fin insertion} (a = 11.4), \emph{eye size} (a = 9), \emph{pectoral fin position} (a = 7.1) and food acquisition through \emph{gill raker type} (a = 17.8) and \emph{oral gape axis} (a = 12,3) (Table \ref{table:cont_abs}).

PC3 --> Locomotion
linked to the locomotion, with strong influence of \emph{anus position} (a = 23.8), \emph{gill raker type} (a = 19.2), \emph{dorsal fin insertion} (a = 15.5) and \emph{pectoral fin position} (a = 14.2) (Table \ref{table:cont_abs}). On the left of this axis, species are characterized by a rather high pectoral fin insertion and close-to-head dorsal fin insertion and anus position (see Figures \ref{fig:famd34} and \ref{fig:corr_circ_34}).

PC4 --> Prey selection
 related to traits linked to prey selection, with \emph{pyloric caeca} (a = 31.6), \emph{operculum volume} (a = 17) and \emph{presence of photophores} (a = 10.9) (Table \ref{table:cont_abs}). The remaining PC scores (20\% of variance) are mainly linked to food acquisition. 

 --> Dimorphisme pas pris en compte mais peut jouer sur la morpho --> Geidner 




Numerous studies have
also confirmed that mouth size and oral anatomy influence the trophic
niche of species (e.g., Wainwright and Richard, 1995; Colborne et al.,
2013). --> PC1

Previous studies showed that functional niche overlap and species abundances were linked \citep{aneeshkumar2017,mason2008}. More precisely, the most abundant species of an ecosystem tend to be functionnally separated from the others \citep{farre2016}. One explanation is that less-abundant (or rare) species have highly specialized morphological features, which may narrow the range of ressources available to use \citep{aneeshkumar2017}. In this ecosystem, Myctophidae, which dominates in terms of abundance and diversity \citep{catul2011,garcia2021}, seem to share similar functional traits. Moreover, our results showed similarities between Myctophidae and Platytrocidae, the two families overlapping each other niches. --> Vrai pour A. olfersii (voir Eduardo2020) --> l'une des familles les + visibles de ces écsosystèmes
In terms of abundance and biomass, representatives of the family
Sternoptychidae (hatchetfishes) are one of the most conspicuous components
of the mesopelagic ichthyofauna (Gjøsaeter and Kawaguchi,
1980).
Other species showing highest abundance (% of species numbers in parentheses) west of the ridge include: Protomyctophum arcticum (99%), Gonostomatidae (98%, mostly damaged Cyclothone),  Lampanyctus macdonaldi (94%), Holtbyrnia anomala (92%), Serrivomer beanii (84%), Malacosteus niger (82%), and Stomias boa ferox (70%). --> 
\citep{sutton2008}


% Impact de la taille dr la niche sur les relations trophiques (compet)
Faire le lien entre la taille de la niche et la compétition : + la niche est petite, plus il y a de proba que la compétition en soit la raison 

% Methode du boostrap et paramétrisation de l'exp
Zhao 2014 --> mêmes conclusions sur l'augmentation de la taille de la niche en fct du nbe d'individus.

communautaire : certaines espèces sont-elles redondantes ? Plusieurs espèces
occupent la même niche en cas de chevauchement, occupent la même niche fonctionnelle. Se focaliser
sur les fonctions plutôt que sur l'espèce. 
		- Individus appartenant à la meme niche peuvent être considérés comme appartenant
à la même boîte fonctionelle. 


Discussion: niches trophiques, comportements, habitats, sensibilité des espèces, 
particularité des espèces, voir les avantages de partager les niches 
