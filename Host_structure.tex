%!TEX root = ./Structure_rapport_final.tex

The Centre d'Etude Biologique de Chizé (CEBC - UMR 7372) is a joint research unit, interested in the adaptation processes of species to climate change with a view to establishing conservation programmes. Different ecosystems are studied by three research teams. The first team focuses on the mechanisms of adaptation in response to environmental variables. The second team is studying the impact of land use changes due to increased agricultural needs on the demography and spatial distribution of vertebrates. Finally, the last team is interested in the impact of climate change on marine mammal and bird populations, particularly in the polar regions. Since 2014, this "Marine Predators" research team has been working together with the PELAGIS observatory, so that the data collected by the observatory can be integrated into international management plans for marine biodiversity.  

Pelagis observatory (UMS 3562) is a joint unit of the CNRS and La Rochelle University, in partnership with the Ministry of Ecology. Specialized in marine mammals and birds monitoring, PELAGIS play an crucial role for biodiversity conservation. Indeed, PELAGIS is responsible for the demographic monitoring of marine mammals and birds in French waters and is in charge of the National Stranding Network. The presence of numerous correspondents throughout the coastline allows rigorous monitoring of strandings. Observation campaigns at sea, whether visual, acoustic or telemetric, provides information on the abundance and distribution of animals. This monitoring is essential for assessing conservation objectives, particularly for species on the IUCN red lists. General health of marine mammals and birds populations are highly dependant of the lower trophic levels they rely on, as main source of food and energy. In this context, PELAGIS, in collaboration with Ifremer's campains, are involved in projects to assess health of these lower trophic levels. 

In particular, the SUMMER (SUstainable Management of MEsolpelagic Ressources) aims to increase the knowledge of mesopelagic ecosystems and to contribute to the preservation of process regulating climate and to the mitigation of impacts of climate change. Many partners, including the University of La Rochelle and Ifremer, are thus working together to know more about the currently unknown mesopelagic communities. Main objectives are to evaluate whether and how mesopelagic resources can be exploited, while quantifying and preserving the ecosystem services the mesopelagic community provide.