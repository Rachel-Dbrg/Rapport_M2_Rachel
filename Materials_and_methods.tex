%!TEX root = ./Structure_rapport_final.tex


\subsection{Sampling and specimens}

Fishes were collected during Ifremer's EVHOE (EValuation Halieutique de l'Ouest Européen) research cruises, surveying the Bay of Biscay every fall onboard the \textit{R/V Thalassa}.
Several hauls are regularly performed at night to investigate pelagic deep fauna. Each station is precisely defined with its GPS coordinates and located above canyons, at the edge of continental shelf. 
% Altough they are considered to be ``biodiversity hotspot'', canyons communities are yet relatively unknown, because of the logistic and material difficulties that their exploration implies \citep{gillet2013}.
Pelagic trawling is performed during night, between 700 and 2000 meters, because those fishes perform diel vertical migrations and tends to come closer to the surface at nighttime. To this end, 25 meters-wide opening trawl is used, with as mesh size decreasing from 76mm to 48mm at the end of the trawl. The trawl-haul duration was 1 hour at 4 kn. 
Once the trawl is pulled back onboard, fishes are sorted, identified, up to species, and frozen at -20°C. Eleven of the most abundant species in the Bay of Biscay have been selected for this study. All teleost, four of them belong to the Myctophid family (lanternfish), which is the most abundant and widely spread family across all oceans \citep{debusserolles2014} and could represent up to 65\% of the pelagic deep-sea biomass \citep{poulsen2013}: \textit{Lampanyctus crocodilus}, \textit{Myctophum punctatum}, \textit{Notoscopelus kroeyeri} and \textit{Ceratoscopelus maderensis}. The second most represented family is the Platytroctidae with two species: \textit{Searsia koefoedi} \& \textit{Normichthys operosus}. This family seems to be found in all oceans but not in Mediterranean Sea \citep{orrell2016}. Finally, five families are represented by one species each: \textit{Xenodermichthys copei} (Alepocephalidae); \textit{Arctozenus risso} (Paralepididae), \textit{Argyropelecus olfersii} (Sternoptychidae - Hatchetfishes), \textit{Serrivomer beanii} (Serrivomeridae) and \textit{Stomias boa} (Stomiidae) which are common species, found abundantly in every ocean \citep{carvalho1988,froese2019,geidner2008,germain2019}. See \ref{fig:phylotree} for phylogenetic tree of studied species. 


\subsection{Morphological measurements and functional traits}
In lab, fish were thaw and 24 morphological measurements were taken on individuals, using an electronic caliper with a precision up to 0.01mm. Part of these measurements had previously been taken by students from La Rochelle University during practictal class from 2018 (\textit{n=99}), 2019 (\textit{n=45}), 2020 (\textit{n=9}) and the rest during this study (\textit{n=212}). For the sake of statistical robustness and representativity, at least 25 individuals were measured for each species (Table \ref{table:spcount}).  

%Voir script table_tex.Rnw
\begin{table}[ht]
\centering
\caption[Count, size's mean and range values of species]{Number of individuals measured for each species, with their mean size and range values}
\label{table:spcount}
\begin{adjustbox}{max width=1.1\textwidth,center}
\begin{tabular}{lrrr|rl}
  \hline
Species & Previous measurements from students & Data acquired during this study & Total & Mean standard length (mm) & Size range (mm) \\ 
  \hline
Lampanyctus crocodilus &  39 &   0 &  39 & 107.60 & 73.3 - 146.5 \\ 
  Xenodermichthys copei &  38 &   0 &  38 & 109.68 & 82.3 - 132 \\ 
  Normichthys operosus &   0 &  38 &  38 & 104.69 & 75.64 - 131.62 \\ 
  Argyropelecus olfersii &  37 &   0 &  37 & 56.55 & 32.16 - 89.07 \\ 
  Notoscopelus kroyeri &   6 &  30 &  36 & 76.30 & 52.63 - 130.84 \\ 
  Searsia koefoedi &   5 &  31 &  36 & 119.94 & 84.8 - 142.75 \\ 
  Arctozenus risso &  20 &  10 &  30 & 158.36 & 117.6 - 181.31 \\ 
  Serrivomer beanii &   0 &  30 &  30 & 546.17 & 373 - 879 \\ 
  Ceratoscopelus maderensis &   0 &  30 &  30 & 62.72 & 53.29 - 78.95 \\ 
  Stomias boa &   0 &  26 &  26 & 239.00 & 144 - 311 \\ 
  Myctophum punctatum &   8 &  17 &  25 & 65.48 & 52.53 - 80.14 \\ 
   \hline
\end{tabular}
\end{adjustbox}
\end{table}


Following what had been made on similar studies and according to our measurements, a total of 21 functional traits were calculated from morphological measurements and informs on 3 main functions: food acquisition, locomotion and habitat (see Table \ref{table:functraits}).

\begin{sidewaystable}
\centering
\caption[Functional traits descriptions and formulas]{Description and formulas of the functionals traits computed from morphological measurements, following \citep{albouy2011, aneeshkumar2017,boyle2006,brindamour2016,diderich2006,dumay2004,habib2019,ibanez2007,sibbing2000,webb1984,winemiller1991}. Abbreviations used in formulas are provided by raw measurements and detailed in appendices \ref{fig:full_body}, \ref{fig:head} \& \ref{fig:fin}. \textsc{oga}, \textsc{git}, \textsc{pc}, \textsc{pht} are categorial variables directly provided by raw measurements with first two scores detailed respectively in appendices \ref{fig:git} \& \ref{fig:oga}.}
\label{table:functraits}
\begin{tabular}{>{\bfseries}lll>{\scshape}l}
  \hline
Function & Functional trait & Description & Formula  \\ 
  \hline
\multirow{13}{*}{Feeding} &Oral gape axis & Feeding position and depth in the water column & oga \\ 
  &Eye size & Detection of preys and visual acuity for predators & ed/hd \\ 
  &Orbital length & Preys size and behavior (buried, camouflaged) & ed/sl \\ 
  &Oral gape surface & Type and size of preys & mw*md / bw*bd \\ 
  &Oral gape shape & Strategy to capture prey & md/mw \\ 
  &Oral gape position & Fedding position in the water column & mo/hd \\ 
  &Lower jaw length & Compromise between power and opening speed of the mouth & ljl/sl \\ 
  &Gill raker type & Filtration capacities of fish & git \\ 
  &Gill outflow & Succion capacities of fish & ow \\ 
  &Head length & Maximum prey size & hl/sl \\ 
  &Pyloric caeca & Presence/Absence of pyloric caeca & pc \\ 
  &Anus position & Digestive tract length & pal/sl \\ 
  \hline
  \multirow{6}{*}{Locomotion} & Body depth & Swimming capacities of fish linked to their food prospection behavior & bd/sl \\ 
  &Pectoral fin position & Maneuvrability of fish & pfi/pfb \\ 
  &Pectoral fin insertion & Maneuvrability of fish & ppl/sl \\ 
  &Transversal shape & Position in the water column and hydrodynamism & bd/bw \\ 
  &Caudal throttle width & Swimming strategy (cruiser/sprinter) and endurance & cpd \\ 
  &Dorsal fin insertion & Swimming type and behavior & pdl/sl \\ 
  \hline
  \multirow{2}{*}{Habitat} &Eye position & Position in the water column (pelagic/sedentary) & eh/hd \\ 
  &Presence photophores & Presence/Absence of photophores (chercher à quoi ça sert) & pht \\ 
  &Operculum volume & Filtering capacity and oxygen captation & od/ow \\ 
   \hline
\end{tabular}
\end{sidewaystable}

\subsection{Data analysis}
All data analysis were performed using \textsf{R} version 4.0.3 \citet{rcoreteam2021}.

\subsubsection{Data pre-processing}
Because measurements came from several observers, raw data had to be checked for outliers. To do so, all values (except \textsc{sl}) were standardized by \textsc{sl} to get rid and the interquartile range (IQR) method of outlier detection was applied to remove outliers. According to this method, for each variable (measurement) and for each species, outliers are defined as every value outside this interval: 
\begin{center}
$ [Q1 - 1.5*QR, Q3 + 1.5*IQR]$ \\
with $Q1$ and $Q3$ being respectively first and third quartile, $IQR = Q3 - Q1$. 
\end{center}{}


Missing values that were previously present in the data set (\textit{n=52}) and those induced by outlier removal function (\textit{n=307}), are then imputed following k-Nearest Neighbor (kNN) method. Based on non-missing values points, kNN algorithm computes Euclidian distances between points, and assumes that missing value can be approximated by the values of points that are the closest. To this end \emph{tidymodels} R package and \emph{step\_knnimpute} function \citep{kuhn2020} were used, with a number of nearest neighbors (\textit{k}) of $\sqrt{N}$, with $N$ being the number of observations, i.e individuals, of the dataset. Accuracy was then checked with linear regression of each variable with \textsc{sl}. From this cleaned dataset, functional traits were computed using formulas in Table \ref{table:functraits}. 

\subsubsection{Factorial Analysis of Mixed Data (FAMD)}
To assess similarity or differences in species in terms of traits, FAMD was performed. This type of analysis is suited for dataset containing a mix of qualitative and quantitative variables, and performs PCA (Principal Component Analysis) on quantitative variables and MCA (Multiple Correspondence Analysis) on qualitative variables. In the same way that PCA, FAMD is used to assess similarity/difference of individuals and to check for association between the variables. To this end, \emph{FactoMineR} package was used. Because \textit{Serrivomer beanii} does not have pectoral nor pelvic fins, and because this analysis can not handle missing values, zero values were attributed for \textsc{PFB}, \textsc{PFI}, \textsc{PPL} and \textsc{PVL}. Because of this specific particularity, complementarty analysis was run. First, same FAMD analysis was run without \textit{S. beanii}, to assess discrimination between all remaining species. Then, the analysis was run without previously cited traits, to look for their weight and to verify that observed differences were not exclusively dependants of these traits. Categorical variables were managed as follow: scores were used for both oral gape axis and gill raker type (see Figures \ref{fig:oga} and \ref{fig:git}), whereas pyloric caeca and photophores were binary defined (0 = absent, 1 = present). For each principal component (PC), trait that have correlation superior or equal to threshold is consider significant, threshold (t) being define by $t = 100 / \text{number of variables}$. Here, 21 functional traits were computed, meaning that $t \approx 4.8$. The drawn ellipse assumes multivariate distribution, at 95\% confidence interval.

 \subsection{Functional niche analysis}
 Coordinates of projected points in the FAMD factorial first and second dimensions were used to compute functional niche surfaces and ellipse overlap. This analysis was only run on the two first PC, because they are the one carrying the more information on the community's structure. Our process being similar to isotopic niche analysis, which is much more documented, we used \emph{Stable Isotope Bayesian Ellispes in R} (SIBER) package and adapted it to our purpose \citep{siber2011}. Species pair-comparisons were performed, for which both ellipse surface and overlap between ellipses were computed using \emph{maxLikeOverlap} function. Absolute niche surfaces were then standardized by the smallest niche, to give relative difference in size niches. Relative overlap was assess by dividing the overlap value by the size of niche, for both species compared. Finally, total overlap was determined by dividing the overlap value by the sum of both compared species niche surface. Niche surface was also estimated for various sample size (\textit{n}), with bootstraped method. To this end, 100 to 10000 individuals of our data set were randomly sampled, and niche's surface was computed. This process has been repeated 10000 times.

 Functional niches were further investigated using \emph{funrar} R package, to compute functional rarity \citep{matthiasgrenie2017}. Based on \citet{violle2017} indices characterizing rarity of functions within a system, the aim is to assess local functional diversity of traits in our dataset. To this end, we first computed a distance matrix, using Gower's distance as it can deal with both quantitatives and qualitatives variables \citep{brindamour2016,matthiasgrenie2017}. On this matrix was applied \emph{distinctiveness global} function, that measures how functionnally rare a species is, comparing to other species of the community. Distinctiveness results range from 0 to 1 (where the species is, locally, functionnally distant/different from the other) \citep{matthiasgrenie2017}.

 \subsection{Kernel density estimation}
 In case of niche overlapping, it is interesting to detail the traits that are most similar between species. To this end, we used \citep{mouillot2005} parametric kernel density function, which has been used in a similar context by \citet{aneeshkumar2017}. Along a trait axis, the function returns the overlap value between $N$ overlapping species, ranging from 0 (no overlap) to 1 ($N$ species have identical densities).