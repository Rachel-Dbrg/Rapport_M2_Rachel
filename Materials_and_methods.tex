%!TEX root = ./Structure_rapport_final.tex


\subsection{Sampling and specimens}

Fishes were collected during Ifremer's EVHOE (EValuation Halieutique de l'Ouest Européen) research cruises, surveying the Bay of Biscay every fall onboard the \textit{R/V Thalassa}. \textbf{Préciser les années}
Several hauls are regularly performed at night to investigate pelagic deep fauna. Each station is precisely defined with its GPS coordinates and located above canyons, at the edge of the continental shelf. 
% Altough they are considered to be ``biodiversity hotspot'', canyons communities are yet relatively unknown, because of the logistic and material difficulties that their exploration implies \citep{gillet2013}.
Pelagic trawling is performed at night, between 700 and 2000 meters, because those fishes perform diel vertical migrations and tend to come closer to the surface at nighttime. To this end, a 25 meters-wide opening trawl is used, with a mesh size decreasing from 76mm to 48mm at the end of the trawl. The trawl-haul duration was 1 hour at 4 kn. 
Once the trawl is pulled back onboard, fishes are sorted, identified up to the species level, and frozen at -20°C. Eleven of the most abundant species in the Bay of Biscay (all teleosts) have been selected for this study. Four of them belong to the Myctophid family (lanternfish), which is the most abundant and widespread family across all oceans \citep{debusserolles2014} and could represent up to 65\% of the pelagic deep-sea biomass \citep{poulsen2013}: \textit{Lampanyctus crocodilus}, \textit{Myctophum punctatum}, \textit{Notoscopelus kroeyeri} and \textit{Ceratoscopelus maderensis}. The second most represented family is the Platytroctidae with two species: \textit{Searsia koefoedi} \& \textit{Normichthys operosus}. This family seems to be found in all oceans but not in the Mediterranean Sea \citep{orrell2016}. Finally, five families are represented by one species each: \textit{Xenodermichthys copei} (Alepocephalidae); \textit{Arctozenus risso} (Paralepididae), \textit{Argyropelecus olfersii} (Sternoptychidae - Hatchetfishes), \textit{Serrivomer beanii} (Serrivomeridae) and \textit{Stomias boa} (Stomiidae) which are common species, found abundantly in every ocean \citep{carvalho1988,froese2019,geidner2008,germain2019}. See \ref{fig:phylotree} for a phylogenetic tree of these species. 


\subsection{Morphological measurements and functional traits}
In the laboratory, the fish were thawed and 24 morphological variables were measured, using an electronic caliper with a precision of 0.01mm. Some of these measurements had previously been recorded by students from La Rochelle University during practical classes in 2018 ($n = 99$), 2019 ($n = 45$), 2020 ($n = 9$) and the rest during this study ($n = 212$). For the sake of statistical robustness and representativity, at least 25 individuals were measured for each species (Table~\ref{table:spcount}).  

%Voir script table_tex.Rnw
\begin{table}[ht]
\centering
\caption[Count, size's mean and range values of species]{Number of individuals measured for each species ($n_1$: previous studies, $n_2$: this study, $n_\text{Tot}$ : $n_1 + n_2$), with their mean standard size and range values in millimeters.}
\label{table:spcount}

\begin{tabular}{rrrrrr}
  \toprule
        &                  &          &       & Mean length & Size range\\ 
Species & $n_1$  & $n_2$ & $n_\text{Tot}$ &   (mm) & (mm) \\ 
  \midrule
  \emph{Lampanyctus crocodilus} &  39 &    &  39 & 107.60 & 73.3 - 146.5 \\ 
  \emph{Xenodermichthys copei} &  38 &    &  38 & 109.68 & 82.3 - 132 \\ 
  \emph{Normichthys operosus} &    &  38 &  38 & 104.69 & 75.64 - 131.62 \\ 
  \emph{Argyropelecus olfersii} &  37 &    &  37 & 56.55 & 32.16 - 89.07 \\ 
  \emph{Notoscopelus kroyeri} &   6 &  30 &  36 & 76.30 & 52.63 - 130.84 \\ 
  \emph{Searsia koefoedi} &   5 &  31 &  36 & 119.94 & 84.8 - 142.75 \\ 
  \emph{Arctozenus risso} &  20 &  10 &  30 & 158.36 & 117.6 - 181.31 \\ 
  \emph{Serrivomer beanii} &    &  30 &  30 & 546.17 & 373 - 879 \\ 
  \emph{Ceratoscopelus maderensis} &    &  30 &  30 & 62.72 & 53.29 - 78.95 \\ 
  \emph{Stomias boa} &    &  26 &  26 & 239.00 & 144 - 311 \\ 
  \emph{Myctophum punctatum} &   8 &  17 &  25 & 65.48 & 52.53 - 80.14 \\ 
  \bottomrule
\end{tabular}
\end{table}

21 functional traits were calculated for each individual from the morphological measurements. They inform on 3 main functions: food acquisition, locomotion and habitat (see Table~\ref{table:functraits}).

\begin{sidewaystable}
\centering
\caption[Functional traits descriptions and formulas]{Description and formulas of the functionals traits computed from morphological measurements, following \citep{albouy2011, aneeshkumar2017,boyle2006,brindamour2016,diderich2006,dumay2004,habib2019,ibanez2007,sibbing2000,webb1984,winemiller1991}. Abbreviations used in formulas are provided by raw measurements and detailed in appendices \ref{fig:full_body}, \ref{fig:head} \& \ref{fig:fin}. \texttt{oga}, \texttt{git}, \texttt{pc}, \texttt{pht} are categorial variables directly provided by raw measurements with the first two scores detailed respectively in appendices \ref{fig:git} \& \ref{fig:oga}.}
\label{table:functraits}
\begin{tabular}{>{\bfseries}llll}
\toprule
Function & Functional trait & Description & Formula  \\ 
\midrule
\multirow{13}{*}{Feeding} &Oral gape axis & Feeding position and depth in the water column & oga \\ 
  &Eye size & Detection of preys and visual acuity for predators & ed/hd \\ 
  &Orbital length & Preys size and behavior (buried, camouflaged) & ed/sl \\ 
  &Oral gape surface & Type and size of preys & mw*md / bw*bd \\ 
  &Oral gape shape & Strategy to capture prey & md/mw \\ 
  &Oral gape position & Fedding position in the water column & mo/hd \\ 
  &Lower jaw length & Compromise between power and opening speed of the mouth & ljl/sl \\ 
  &Gill raker type & Filtration capacities of fish & git \\ 
  &Gill outflow & Succion capacities of fish & ow \\ 
  &Head length & Maximum prey size & hl/sl \\ 
  &Pyloric caeca & Presence/Absence of pyloric caeca & pc \\ 
  &Anus position & Digestive tract length & pal/sl \\ 
\midrule
  \multirow{6}{*}{Locomotion} & Body depth & Swimming capacities of fish linked to their food prospection behavior & bd/sl \\ 
  &Pectoral fin position & Maneuvrability of fish & pfi/pfb \\ 
  &Pectoral fin insertion & Maneuvrability of fish & ppl/sl \\ 
  &Transversal shape & Position in the water column and hydrodynamism & bd/bw \\ 
  &Caudal throttle width & Swimming strategy (cruiser/sprinter) and endurance & cpd \\ 
  &Dorsal fin insertion & Swimming type and behavior & pdl/sl \\ 
\midrule
  \multirow{2}{*}{Habitat} &Eye position & Position in the water column (pelagic/sedentary) & eh/hd \\ 
  &Presence photophores & Presence/Absence of photophores (chercher à quoi ça sert) & pht \\ 
  &Operculum volume & Filtering capacity and oxygen captation & od/ow \\ 
\bottomrule
\end{tabular}
\end{sidewaystable}

\subsection{Data analysis}
All data analyses were performed using \textsf{R} version 4.0.3 \citep{rcoreteam2021}.

\textbf{Commentaire : } je vois que tu utilises \verb;\texttsc{}; quand tu parles de tes mesures et des traits. Attention, je crois que le Times New Roman ne dispose pas de cette police spéciale qui fait des "Petites majuscules". Du coup, la commande n'a aucun effet. J'ai donc remplacé par \verb;\texttt; partout où je l'ai trouvé ! D'ailleurs, utilises aussi plutôt ça plutôt que l'italique pour les noms de logiciels, de packages, etc.

\subsubsection{Data pre-processing}
Because measurements came from several observers, raw data had to be checked for outliers. To do so, all values (except the standard length \texttt{sl}) were standardized by \texttt{sl} and the interquartile range (IQR) method of outlier detection was applied to remove outliers. According to this method, for each measurement and species, outliers are defined as every value outside this interval: 
\begin{center}
$ [Q1 - 1.5 \times IQR, Q3 + 1.5 \times IQR]$ \\
with $Q1$ and $Q3$ being respectively the first and third quartile, and $IQR = Q3 - Q1$. 
\end{center}{}

Missing values initially present in the dataset ($n = 52$) and induced by the outlier removal function ($n = 307$), were then imputed using the $k$-Nearest Neighbor (kNN) algorithm. Suppose that the individual $i$ has a missing value $X_i$ for the variable $X$. The algorithm identifies the $k$ nearest neighbors of $i$ based on all available variables (\emph{i.e.} all variables but $X$). The values of $X$ observed in the $k$ nearest neighbors of $i$ are then used to infer $X_i$ using Euclidean distances between $i$ and its $k$ nearest neighbors. The computations were performed by the \texttt{step\_knnimpute()} function from the \texttt{tidymodels} R package \citep{kuhn2020}, with a number of nearest neighbors of $k = \sqrt{N}$, with $N$ being the number of individuals for each species. The accuracy of the imputation was then checked with a linear regression of each variable with respect to \texttt{sl}. From this cleaned dataset, the functional traits were computed using formulas in Table~\ref{table:functraits}. 

\subsubsection{Factorial Analysis of Mixed Data (FAMD)}
To assess the similarities or differences of traits in species, a Factor Analysis of Mixed Data (FAMD) was performed using the \texttt{FactoMineR} package. This type of multivariate analysis is suited for datasets containing a mix of qualitative and quantitative variables. It performs a Principal Component Analysis (PCA) on quantitative variables and a Multiple Correspondence Analysis (MCA) on qualitative variables. The results of an FAMD are interpreted in the same way as those of a PCA: both methods are used to assess (i) similarities and differences among individuals and (ii) to check for associations between variables. Because \textit{Serrivomer beanii} does not have pectoral nor pelvic fins, and because this analysis can not handle missing values, zero values were attributed for \texttt{PFB}, \texttt{PFI}, \texttt{PPL} and \texttt{PVL}. Because of this specific particularity, complementarty analysis was run. First, the same FAMD analysis was run without \textit{S. beanii}, to assess the discrimination between all remaining species. Then, the analysis was run with all species (including \emph{S. beanii}) but without the traits missing for \emph{S. beanii} (\emph{i.e.} \texttt{PFB}, \texttt{PFI}, \texttt{PPL} and \texttt{PVL}), to verify that the observed differences were not exclusively dependants on these traits. The categorical variables were managed as follow: scores were used for both oral gape axis and gill raker type (see Figures~\ref{fig:oga} and \ref{fig:git}), whereas pyloric caeca and photophores were binary defined (0 = absent, 1 = present). For each principal component (PC), all traits that have a contribution greater than or equal to a threshold $t$ are considered relevant (with $t = 100 / \text{number of variables}$). Here, 21 functional traits were computed, meaning that $t \approx 4.8$. On the resulting FAMD graph, we added inertia ellipses representing the functional niche of each species. These ellipses assume a multinormal distribution of the functional traits.

 \subsection{Functional niche analysis}
 The coordinates of individuals projected on the FAMD axes were used to compute (i) the surface of the functional niche of each species, and the overlap of niches for all possible pairs of species. This analysis was only run on the first two factorial axes (PCs), because they are the ones carrying most of the information about the community's structure. The method used here is similar to isotopic niche analysis, which is much more documented. We used the \texttt{SIBER} package (Stable Isotope Bayesian Ellispes in R) and adapted it to our purpose \citep{siber2011}. For each species the ellipse surface was computed, and the overlap of ellipses between all pairs of species was assessed with the \texttt{maxLikeOverlap()} function. Absolute niche surfaces were then standardized by the smallest niche, to facilitate the comparison of relative differences in niche sizes. The relative overlap was assessed as the ratio of the overlap value to the size of the niche, for each species of each pair. Finally, the total overlap for a pair of species was determined by dividing the overlap value by the sum of the area of the niches of the two species. In order to assess the sensitivity of the method to insufficient sample size, niche surfaces were also computed for various sample sizes ($n$) following a bootstrap procedure. To this end, between 100 to $10\,000$ individuals from our dataset were sampled at random, and area of all niche's was recalculated $10\,000$ times.

 Functional niches were further investigated using the \texttt{funrar} R package, to compute functional rarity \citep{matthiasgrenie2017}. Based on indices characterizing the rarity of functions within a system \citep{violle2017}, the aim is to assess the local functional diversity of the traits in our dataset. First a distance matrix is computed using Gower's distance as it can deal with both quantitatives and qualitatives variables \citep{brindamour2016,matthiasgrenie2017}. Then, the \texttt{distinctiveness\_global()} function is applied to measure how functionnally rare a species is, compared to other species of the community. Distinctiveness results range from 0 (no distinctiveness) to 1, when a species is, locally and/or functionnally, different from the others \citep{matthiasgrenie2017}.

 \subsection{Kernel density estimation}
 When two or more niches overlap, it is interesting to detail which traits are most similar between species. To this end, we used a parametric kernel density function \citep{mouillot2005} , which has been used in a similar context by \citet{aneeshkumar2017}. Along a trait axis, the function returns the overlap value between $N$ overlapping species, ranging from 0 (no overlap) to 1 ($N$ species have identical, fully overlapping densities).