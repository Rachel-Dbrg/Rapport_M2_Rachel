%!TEX root = ./Structure_rapport_final.tex


\subsection{Structure and dynamic of ecosystems: how can species coexist in same environment?} 

\emph{Comment: sauf cas rare, le mot "dynamics" s'écrit (presque) toujours au pluriel.}

Of all the questions raised when it comes to study Nature, the most common yet complex one is ``how do organisms and environment interact with each other?'' \citep{sutherland2013}. In other words, what are the processes and rules that define the structure and functioning of ecosystems? Taken as a whole, an ecosystem can be seen as a giant network: all individuals from every species are linked to one another through intra- and interspecific relationships, that implies competition, parasitism, predation\ldots{}; and each individual is linked to its physical environment, on which it depends for food prospection, shelter and/or favourable conditions for breeding. The main purpose of ecology is to study those links and ultimately, to be able to map the central relationships and flows that are keys to maintain stable ecosystems. Community ecologists, regardless of the ecosystems they are studying, try to answer some fairly similar questions such as ``how do species share environmental resources'' or ``how can species relying on the same resource to thrive and survive can coexist?''. Indeed, the mere observation that species can live and develop a population without encroaching each other suggests that, even if species compete to access the same resources, the use of resources is balanced and allows stable relationships between species to develop. Most of all, if species depend on the same resources to survive, how can diversity within an ecosystem last over time?

\subsection{Concepts and definition for studying ecosystems’ dynamic.}
 
To answer those questions, structure and dynamic of ecosystems needs to be investigated and ecological concepts must be defined. First, the whole living compartment is define as a “community”, which gather all organisms, thus different species, that live and interact together in a same habitat. Within this habitat, populations might differently use space and food resources they need to survive, that are defined as “niche”. In 1917, Grinnell was the first to scientifically define a term for all the requirements that a species need to thrive with “ecological niche”. This term includes biotic (food abundance and availability for prospection, competition among species, predation-prey relationships) and abiotic (environmental conditions such as temperature or pression, shelter) conditions and shape the areas suited for species according to their needs. This definition is refined by Hutchinson in 1957 : resources and their availability are the main factors that conditions coexistence of species, for resources are essential for species to thrive and are therefore considered limiting factors. Furthermore, Hutchinson’s definition distinguishes the concepts of ‘fundamental niche’, which is a potential niche for a given species that offers all the optimal conditions for that species to flourish and ‘realized niche’, which are the actual resources used by a species and which is often smaller than fundamental niche, mainly because of competition between species. If resource is limited, competition might occurs between species that seek simultaneously for the same resource, whether it is that they hunt the same preys or live in a same specific habitat, because they are more likely to meet \citep{blondel1979}. When the resources are abundant enough, species may be able to share it without entering competition \citep{nagelkerke2018}. According to competition theory, community structures are defined by the way species are able to share or not resources. Therefore, to study the structure and dynamic of community and to identify the main factors that enable species to coexist, it is essential to determine the degree by which species share resources, in other words, quantify the overlapping between niches \citep{geange2011}.


\subsection{Tools to study diversity in ecosystems and their limits}

Niche overlapping can be approached from the habitat (beta overlapping) or food (alpha overlapping) perspective. In both cases, a high degree of overlapping for species means that they are very likely to compete for the same resources and their coexistence is virtually impossible. On the other hand, low overlapping degree tend to suggest that species, even if they depend partly on the same resource, have a wide enough range of accessible resources not to compete with each other \citep{mouillot2005}. No overlapping means that species are likely to use very different resources, and that no competition is expected. This approach has been greatly developed since few decades, with the urge to study the abundance distribution and the mechanisms supporting species coexistence, in order to predict the impact of disturbances, such as the introduction of invasive species or climate change \citep{albouy2011,geange2011, martini2020}. To quantify niche overlapping, several indices have been developed since the 60’s. Among them, four of the most famous ones, relying on the intensity of utilisation of a resource by species, have been compared by Linton, Davies, and Wrona, 1981) to assess the precision and accuracy of each one: Morisita (1959) updated by Horn (1966), Schoener (1968) and Pianka (1973). Index and threshold values are commonly used for studying diversity, whether it is specific, taxonomic, or phylogenetic, but used alone, a result of diversity estimation through any index is usually poor and inaccurate, because complex and rich systems can not be described only by the result of a computation \citep{mejri2009}. Overlapping indices are no exception to the rule: even if they generally lead to the same conclusion, the four indices previously mentioned often give different results, because they use different computation parameters \citep{blondel1979}. As such, they provide a qualitative assessment of the overlapping rather than a quantitative one. On top of that, the results given by those indices are highly dependent of sample size, which adds uncertainty when it comes to conclude from the computation result \citep{linton1981}. Finally, \citep{grossman2009} points out that threshold values for those indices can be considered as arbitrary and might differ from one ecosystem to another, leading to an impossibility of comparing them. For all those reasons, using those indices to estimate if species share or not the same resources, and if so, how much is shared, does not seem relevant \citep{mouillot2005}. 
Therefore, to understand how the structure and the dynamic of an ecosystem are defined and how such complex relationships can last for several generations, simulation through models are often used (see Ecopath =\url{https://ecopath.org}). This approach requires a simplification of the ecosystem, for too complex models are virtually impossible to compute \citep{albouy2011}. Simplifying an ecosystem can be done in many ways: focusing on specific aspects of the ecosystem (for instance: pelagic or benthic fauna), adopting a trophic strategy (gather species according to their trophic level) or gathering species that are close in terms of taxonomy or behave alike. Obviously, simplifying with any of those methods ends up with approximating the relationships and much of the complexity of an ecosystem, but if done properly, models are still able to display reliable simulation of what is going on in real life. Yet, the main difficulty is to determine the criteria that are relevant to gather species and to simplify models. Whether they are too or not selective enough, those criteria not only condition the accuracy of the model, but also its capacity of being generalised. For instance, if taxonomic criteria are used to gather species to model a given ecosystem, this model will only be suited to study ecosystems that contains those species. Thus, transposition of a taxonomic-based model for other ecosystems studies is quite limited, if not impossible. Furthermore, this method requires that each ecosystem should have a specific model which is highly time-consuming and compromise comparisons between ecosystems \citep{martini2020, mcgill2006}. For all these reasons, tools used are highly specific of the studied ecosystem and of the species in it. 

%How does the morphology of fish impact its behaviour? 
%How can the morphology of the fish help us to understand more about its behaviour?

\subsection{Emergence of a more global approach based on functional-traits}

\subsubsection{General overview}
Community ecology’s aim to establish generalised rules to explain how communities works, and species-based approach only provides information for a few specific systems but not global principles that can be applied to the whole community. Therefore, ecologists needed to find a way to study ecosystems, that could, on one hand, give clues of how species interacts with each other and, in the other hand, translate to which degree species are linked to their environment. Indeed, some scientists emphasised the urge to get rid of methods that were highly dependent of species, time or space, such as the ones described previously, and to use a more predictable and quantitative science that could play a major role in assessing global changes issues. To this extend, \citep{mcgill2006} suggest that community ecology should rely on traits, in order to understand the way traits and fundamental niche interactions converge toward realised niche. They define a trait as a “well-defined, measurable property of organisms, usually measured at the individual level and used comparatively across species”. Though, when looking for “trait” in bibliography, it appears that this term is widely used, with slightly different meanings. For instance, \citep{violle2007} defined “trait” as “any morphological, physiological or phenological feature measurable at the individual level, from cell to whole organism […]”. To clarify this term and to assure a consistency in community ecology, a recent study suggests that \citep{violle2007}definition is accurate enough to serve as reference and that this term should be used advisedly \citep{martini2020}. Yet, not all measurables traits carries the same information: for ecologists, traits that gives information about the interaction with its environment and the fitness of individuals are the most valuables \citep{kremer2017}. Those specific traits are defined as “functional traits” and can relate to behaviour, life history, morphology or physiology, influencing the general performance of organisms \citep{martini2020, mcgill2006} and inform on majors functions of the organisms, such as food acquisition and locomotion \citep{mejri2009}.

\subsubsection{Improvment of the method over the years}
Developed with studies based on terrestrial plants, functional-traits approach showed that morphology of species was correlated to their environment and that changes in the habitat could lead to changes of species’ morphology, because this approach relies on the plasticity of traits \citep{lavorel1997,martini2020}. Applied to aquatic animals, structure-function relationship has been well documented since 70’s \citep{gosline1971, lagler1977, webb1984} and morphological traits based approach seemed suitable to compare species \citep{norton1995} or to explore niches and to compare communities \citep{winemiller1991}. As it is, \citep{albouy2011} developped a model that was able to determine diet of any marine species based on morphological traits, and thus establish trophic guilds. Yet, the model could not predict diet overlap and resource partitionning between species, because of intrinsic variability of fish diet. Indeed, as pointed out by \citep{sibbing2000}, morphology alone is hardly a clue to determine diet, for generalist species are able to switch preys according to what is more abundant and do not display specific morphological traits. Moreoever, trophic level impacts how specialised a species can be in terms of diet: apex predators will often favor one feeding strategies among others, so they are very efficient for one strategy, and have limitating capacities in others. This principle can be summed up as a ``trade-off strategy'': greater abilities in one area lead to a decrease of abilities in other areas \citep{norton1995}

Conversely, When studying morphological trait associated with swimming performances, \citep{webb1984} noticed that most species were not specialised (\textit{i.e not displaying any particular traits}) and had fairly good performances in 3 of the main swimming methods (powerful-short acceleration, cruise and maneuvrability). 

Same observation has been later made by \citep{grossman2009}, who showed that using only morphological traits to describe majors functions such as food acquisition appears to be not that accurate. To study how communities share food resources, diet from different species with different morphology were compared. Unlike what was expecting, \textit{i.e} species with similar morphology use same resources in the same way, it appears that the link between morphology and food acquisition was not that obvious and robust. Observing that species with high morphological divergences used the same resources in a similar way, whereas morphologically close species displayed very different diet, they conclude that the morphological traits choosen for this study could not fully explain the food acquisition methods. They also make the hypothesis that, to overcome limitations caused by their morphology and/or their habitat when it comes to diet, organisms might modulate their behaviour and show high adaptability, which is a hypothesis shared by \citep{blondel1979}. In fact, if morphology sets limits to resource usage, species usually displays some flexibility around those limits to adapt according to prey availability and environmental conditions \citep{ibanez2007,sibbing2000}. 

If species flexibility and inherent variabiliy must not be ignored, they can be hard to predict. Thus, it is essential to identify and select relevant traits, that can be used to explain most of how species interacts with their environment. This is one of the main obstacles of the functional-traits-based approach, for selected trait must be different enough (\text{i.e} display some variation) between compared levels (species, populations, individuals …) and for observed variation has to explain differences in fitness or coexistence that has been witnessed \citep{kremer2017}. Yet, flexibility in traits is what makes traits-based approach so useful, for it quantifies intraspecific variability, especially when the environmental conditions changes \citep{martini2020} and interspecific variability explaining interactions between species and their environment. In a nutshell, traits that are to be used for functional-trait approach should offer the best trade-off between being informative enough regarding the purpose of the study, generic enough to make it comparable for several species, even if highly morphologically different and also easy to measure and being easily measurable to insure repeatability across studies \citep{dumay2004, kremer2017}.

\subsection{The advent of functional diversity}
Functional approach is relatively recent, and begun to thrive during the 80's with populations collaps and biodversity crisis, through species extinctions \citep{wilson1988}. Indeed, as reported by \citep{mejri2009}, ecologists observed that if a species disapeared from an ecosystem, it did not necessarily meant that the whole ecosystem as troubled or even collapsed. Thus, functions performed by species started to be further investigated, with the need to define role of species within ecosystems. The main question raised by disapearing species not leading to changes in the ecosystem was ``Are all species essential to ecosytems ?''. In other words, can species be considered as ``redundant" if they have a same role in an ecosystem? To answer those questions and to study roles of species within an ecosystem, functional-trait approach seemed relevant, because it gives informations about the roles of species in their environment, which are complementary to those supplied by classical diversity index, such as species diversity, richness distribution or eveness... \citep{marcon2015,mejri2009}. More important, functional-trait and attribution of roles to species is crucial to determine functional diversity, which is the main factor explaining stability and productivity of an ecosystem, and should therefore be preferred to specific or taxonomic diversity when studying community ecology \citep{dumay2004,mejri2009}. In fact, sustainability and prosperity of ecosystems are much more depending of the range of functions and functional-traits displayed by the species, rather than the number of species (\textit{i.e} specific richness) itself. Indeed, abundance and species diversity index relies on the assumption that all species are equivalent, and do not take into account the function provided by those species \citep{mejri2009}. From this approach, richness of an ecosytem is determined by the scope of functional diversity displayed by the species \citep{rocklin2004}. 

To estimate functional diversity, species must first be assigned into ``functional groups'', that defined species similarity with 3 criterias. Species that are in a same functional group must: 
\begin{itemize}
\item share same habitat and same trophic level \citep{brindamour2016}
\item play a similar part, through the functions they provide, in the habitat \citep{dumay2004,mejri2009}
\item display similar responses to changing environmental pressures \citep{dumay2004,brindamour2016,mejri2009}
\end{itemize}

To constitue those groups and evaluate the response of species for each of those 3 items, morphological traits are often used, for they reflect the possibilities and modalities by which species interact with their environment, and can thus be used as indicators of trophic networks or habitats \citep{brindamour2016}. Indeed, according to the “niche filtering hypothesis”, where characteristics of habitats are considered as filters, only species that has the suited traits can thrive in a set of specific environmental conditions \citep{brindamour2011}. In addition, this hypothesis means that species, if they share similar functional traits, must use the same resources, probably in the same ways, hence overlapping each other niches. Conversely, if species display very different morphological traits from one another, they probably use resources in very different ways, or even different resources. 

\subsubsection{Beneficts of this approach}
If several species are gathered into a same functional group, they can be considered as ``functionally equivalent'', with similar or interchanging functions, and the ecosystem they are in thus diplay ``functional redundancy'', which is reducing the risk of a functional loss in case of disruption in the ecosystem. On the other hand, a single species can constitute a group (therefore described as `monospecific') and be considered as ``essential'', for the function provided only relies on the species, and if latter disapears, the associated function will disapear too, leading to severe ecosystem disturbance and affect other major functions \citep{mejri2009}. 
For conservation matters and predictions of how climate change will impact ecosystems and affect biodiversity, functional approach and niche overlapping seems to be relevant, because those tools gives a quantitative overview of resilience and/or resistance of communities or ecosystem facing changes. Not based on species or taxonomy, this approach is more suited for generalisation and estimation of ecosystem services provided \citep{martini2020,mcgill2006}, enhancing our ability to predict ecological dynamic and its fluctuation, in a world facing anthropic influence \citep{kremer2017}.




Descrition des 3 grandes fonctions pour grouper les espèces + mention des performances 

Developpement de nouveaux indices de diversité fonctionnelle 
