%!TEX root = ./Structure_rapport_final.tex

The deep ocean is the largest biome of the Earth, yet, these extrem ecosystems remain mostly unknown, with only 0.001\% of it having been investigated. At the interface between nutrients-rich continental waters to impoverished deep-sea waters, are found submarine canyons. Being highly heterogeneous, these habitats appears to be "biodiversity hotspots" and support many ecological services. Species and communities living in such habitat are understudied and very little is known on resource partitioning supporting co-existence of species and functioning of deep sea ecosystems. In this study, our objectives was to characterise functional diversity through morphology, of 11 common species of Bay of Biscay's canyons, and to measure the degree of overlap bewteen functional niches. Our results revealed significant differences in functional niches between species and in functional traits for species with relatively close niches. Ecological data being scarce on studied species, this work confirms that functional analyses is adapted to deep-sea fishes and that it give informations on the main functions performed by species. Finally, our results supports the hypothesis of high functional diversity in deep-sea ecosystems,  promoting co-existence of species du to high resource partitioning. The present work is a first attempt to compare the morpho-functional characteristics and niche partitioning in the deep-sea fish assemblages from the canyons of Bay of Biscay.\\


Key words: Functional diversity, niche overlap, functional traits, resource partitioning, deep-sea fishes, Bay of Biscay, canyons.

\section{Résumé}
Les eaux profondes constituent le plus grand biome de la Terre, pourtant, ces écosystèmes extrêmes restent pour la plupart inconnus, seuls 0,001\% d'entre eux ayant été étudiés. À l'interface entre les eaux continentales riches en nutriments et les eaux profondes appauvries, on trouve les canyons sous-marins. Très hétérogènes, ces habitats apparaissent comme des "points chauds de la biodiversité" et assurent de nombreux services écologiques. Les espèces et les communautés vivant dans ces habitats sont peu étudiées et l'on sait très peu de choses sur le partage des ressources favorisant la coexistence des espèces et le fonctionnement des écosystèmes d'eaux profondes. Dans cette étude, nos objectifs étaient de caractériser la diversité fonctionnelle, à travers la morphologie, de 11 espèces communes des canyons du Golfe de Gascogne, et de mesurer le degré de chevauchement entre les niches fonctionnelles. Nos résultats ont révélé des différences significatives dans les niches fonctionnelles entre les espèces et dans les traits fonctionnels pour les espèces ayant des niches relativement proches. Les données écologiques étant rares sur les espèces étudiées, ce travail confirme que l'analyse fonctionnelle est adaptée aux poissons d'eaux profondes et qu'elle donne des informations sur les principales fonctions remplies par les espèces. Enfin, nos résultats soutiennent l'hypothèse d'une grande diversité fonctionnelle dans les écosystèmes d'eaux profondes, favorisant la coexistence des espèces grâce à un partage élevé des ressources. Le présent travail est le premire se proposant de comparer les caractéristiques morpho-fonctionnelles et le partage de niche des assemblages de poissons des canyons du Golfe de Gascogne.\\

Mots clés : Diversité fonctionnelle, chevauchement de niche, traits fonctionnels, partitionnement des ressources, poissons d'eau profonde, Golfe de Gascogne, canyons.