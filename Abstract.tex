KUMAR : It is necessary to understand the resource partitioning among species, to describe the functioning of deep sea
marine ecosystems. Functional morphology is an effective approach to understand and compare taxonomic
units with different phenotypic characteristics related to swimming and foraging. In this study, our main
objective was to delineate the functional traits with a view to determine the discrimination level and degree of
functional niche overlap among seven common species inhabiting deep-sea waters in south-eastern Arabian
Sea. Results indicated significant differences in the functional traits between species providing a low functional
niche overlap. Although, the ecological and biological information of fishes were scarce, we demonstrated that
functional analyses are effective to extrapolate the prey preferences, sizes and detection and propulsion
efficiencies for their feeding and swimming strategies. Our study supports the hypothesis that in some
environments with a limitation of resources, species coexisting is due to high resource partitioning. The present
work is a first attempt to compare the morpho-functional characteristics and niche partitioning in the deep-sea
fish assemblages from the Indian waters.